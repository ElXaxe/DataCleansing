
% Default to the notebook output style

    


% Inherit from the specified cell style.




    
\documentclass[11pt]{article}

    
    
    \usepackage[T1]{fontenc}
    % Nicer default font (+ math font) than Computer Modern for most use cases
    \usepackage{mathpazo}

    % Basic figure setup, for now with no caption control since it's done
    % automatically by Pandoc (which extracts ![](path) syntax from Markdown).
    \usepackage{graphicx}
    % We will generate all images so they have a width \maxwidth. This means
    % that they will get their normal width if they fit onto the page, but
    % are scaled down if they would overflow the margins.
    \makeatletter
    \def\maxwidth{\ifdim\Gin@nat@width>\linewidth\linewidth
    \else\Gin@nat@width\fi}
    \makeatother
    \let\Oldincludegraphics\includegraphics
    % Set max figure width to be 80% of text width, for now hardcoded.
    \renewcommand{\includegraphics}[1]{\Oldincludegraphics[width=.8\maxwidth]{#1}}
    % Ensure that by default, figures have no caption (until we provide a
    % proper Figure object with a Caption API and a way to capture that
    % in the conversion process - todo).
    \usepackage{caption}
    \DeclareCaptionLabelFormat{nolabel}{}
    \captionsetup{labelformat=nolabel}

    \usepackage{adjustbox} % Used to constrain images to a maximum size 
    \usepackage{xcolor} % Allow colors to be defined
    \usepackage{enumerate} % Needed for markdown enumerations to work
    \usepackage{geometry} % Used to adjust the document margins
    \usepackage{amsmath} % Equations
    \usepackage{amssymb} % Equations
    \usepackage{textcomp} % defines textquotesingle
    % Hack from http://tex.stackexchange.com/a/47451/13684:
    \AtBeginDocument{%
        \def\PYZsq{\textquotesingle}% Upright quotes in Pygmentized code
    }
    \usepackage{upquote} % Upright quotes for verbatim code
    \usepackage{eurosym} % defines \euro
    \usepackage[mathletters]{ucs} % Extended unicode (utf-8) support
    \usepackage[utf8x]{inputenc} % Allow utf-8 characters in the tex document
    \usepackage{fancyvrb} % verbatim replacement that allows latex
    \usepackage{grffile} % extends the file name processing of package graphics 
                         % to support a larger range 
    % The hyperref package gives us a pdf with properly built
    % internal navigation ('pdf bookmarks' for the table of contents,
    % internal cross-reference links, web links for URLs, etc.)
    \usepackage{hyperref}
    \usepackage{longtable} % longtable support required by pandoc >1.10
    \usepackage{booktabs}  % table support for pandoc > 1.12.2
    \usepackage[inline]{enumitem} % IRkernel/repr support (it uses the enumerate* environment)
    \usepackage[normalem]{ulem} % ulem is needed to support strikethroughs (\sout)
                                % normalem makes italics be italics, not underlines
    

    
    
    % Colors for the hyperref package
    \definecolor{urlcolor}{rgb}{0,.145,.698}
    \definecolor{linkcolor}{rgb}{.71,0.21,0.01}
    \definecolor{citecolor}{rgb}{.12,.54,.11}

    % ANSI colors
    \definecolor{ansi-black}{HTML}{3E424D}
    \definecolor{ansi-black-intense}{HTML}{282C36}
    \definecolor{ansi-red}{HTML}{E75C58}
    \definecolor{ansi-red-intense}{HTML}{B22B31}
    \definecolor{ansi-green}{HTML}{00A250}
    \definecolor{ansi-green-intense}{HTML}{007427}
    \definecolor{ansi-yellow}{HTML}{DDB62B}
    \definecolor{ansi-yellow-intense}{HTML}{B27D12}
    \definecolor{ansi-blue}{HTML}{208FFB}
    \definecolor{ansi-blue-intense}{HTML}{0065CA}
    \definecolor{ansi-magenta}{HTML}{D160C4}
    \definecolor{ansi-magenta-intense}{HTML}{A03196}
    \definecolor{ansi-cyan}{HTML}{60C6C8}
    \definecolor{ansi-cyan-intense}{HTML}{258F8F}
    \definecolor{ansi-white}{HTML}{C5C1B4}
    \definecolor{ansi-white-intense}{HTML}{A1A6B2}

    % commands and environments needed by pandoc snippets
    % extracted from the output of `pandoc -s`
    \providecommand{\tightlist}{%
      \setlength{\itemsep}{0pt}\setlength{\parskip}{0pt}}
    \DefineVerbatimEnvironment{Highlighting}{Verbatim}{commandchars=\\\{\}}
    % Add ',fontsize=\small' for more characters per line
    \newenvironment{Shaded}{}{}
    \newcommand{\KeywordTok}[1]{\textcolor[rgb]{0.00,0.44,0.13}{\textbf{{#1}}}}
    \newcommand{\DataTypeTok}[1]{\textcolor[rgb]{0.56,0.13,0.00}{{#1}}}
    \newcommand{\DecValTok}[1]{\textcolor[rgb]{0.25,0.63,0.44}{{#1}}}
    \newcommand{\BaseNTok}[1]{\textcolor[rgb]{0.25,0.63,0.44}{{#1}}}
    \newcommand{\FloatTok}[1]{\textcolor[rgb]{0.25,0.63,0.44}{{#1}}}
    \newcommand{\CharTok}[1]{\textcolor[rgb]{0.25,0.44,0.63}{{#1}}}
    \newcommand{\StringTok}[1]{\textcolor[rgb]{0.25,0.44,0.63}{{#1}}}
    \newcommand{\CommentTok}[1]{\textcolor[rgb]{0.38,0.63,0.69}{\textit{{#1}}}}
    \newcommand{\OtherTok}[1]{\textcolor[rgb]{0.00,0.44,0.13}{{#1}}}
    \newcommand{\AlertTok}[1]{\textcolor[rgb]{1.00,0.00,0.00}{\textbf{{#1}}}}
    \newcommand{\FunctionTok}[1]{\textcolor[rgb]{0.02,0.16,0.49}{{#1}}}
    \newcommand{\RegionMarkerTok}[1]{{#1}}
    \newcommand{\ErrorTok}[1]{\textcolor[rgb]{1.00,0.00,0.00}{\textbf{{#1}}}}
    \newcommand{\NormalTok}[1]{{#1}}
    
    % Additional commands for more recent versions of Pandoc
    \newcommand{\ConstantTok}[1]{\textcolor[rgb]{0.53,0.00,0.00}{{#1}}}
    \newcommand{\SpecialCharTok}[1]{\textcolor[rgb]{0.25,0.44,0.63}{{#1}}}
    \newcommand{\VerbatimStringTok}[1]{\textcolor[rgb]{0.25,0.44,0.63}{{#1}}}
    \newcommand{\SpecialStringTok}[1]{\textcolor[rgb]{0.73,0.40,0.53}{{#1}}}
    \newcommand{\ImportTok}[1]{{#1}}
    \newcommand{\DocumentationTok}[1]{\textcolor[rgb]{0.73,0.13,0.13}{\textit{{#1}}}}
    \newcommand{\AnnotationTok}[1]{\textcolor[rgb]{0.38,0.63,0.69}{\textbf{\textit{{#1}}}}}
    \newcommand{\CommentVarTok}[1]{\textcolor[rgb]{0.38,0.63,0.69}{\textbf{\textit{{#1}}}}}
    \newcommand{\VariableTok}[1]{\textcolor[rgb]{0.10,0.09,0.49}{{#1}}}
    \newcommand{\ControlFlowTok}[1]{\textcolor[rgb]{0.00,0.44,0.13}{\textbf{{#1}}}}
    \newcommand{\OperatorTok}[1]{\textcolor[rgb]{0.40,0.40,0.40}{{#1}}}
    \newcommand{\BuiltInTok}[1]{{#1}}
    \newcommand{\ExtensionTok}[1]{{#1}}
    \newcommand{\PreprocessorTok}[1]{\textcolor[rgb]{0.74,0.48,0.00}{{#1}}}
    \newcommand{\AttributeTok}[1]{\textcolor[rgb]{0.49,0.56,0.16}{{#1}}}
    \newcommand{\InformationTok}[1]{\textcolor[rgb]{0.38,0.63,0.69}{\textbf{\textit{{#1}}}}}
    \newcommand{\WarningTok}[1]{\textcolor[rgb]{0.38,0.63,0.69}{\textbf{\textit{{#1}}}}}
    
    
    % Define a nice break command that doesn't care if a line doesn't already
    % exist.
    \def\br{\hspace*{\fill} \\* }
    % Math Jax compatability definitions
    \def\gt{>}
    \def\lt{<}
    % Document parameters
    \title{Practica2}
    
    
    

    % Pygments definitions
    
\makeatletter
\def\PY@reset{\let\PY@it=\relax \let\PY@bf=\relax%
    \let\PY@ul=\relax \let\PY@tc=\relax%
    \let\PY@bc=\relax \let\PY@ff=\relax}
\def\PY@tok#1{\csname PY@tok@#1\endcsname}
\def\PY@toks#1+{\ifx\relax#1\empty\else%
    \PY@tok{#1}\expandafter\PY@toks\fi}
\def\PY@do#1{\PY@bc{\PY@tc{\PY@ul{%
    \PY@it{\PY@bf{\PY@ff{#1}}}}}}}
\def\PY#1#2{\PY@reset\PY@toks#1+\relax+\PY@do{#2}}

\expandafter\def\csname PY@tok@w\endcsname{\def\PY@tc##1{\textcolor[rgb]{0.73,0.73,0.73}{##1}}}
\expandafter\def\csname PY@tok@c\endcsname{\let\PY@it=\textit\def\PY@tc##1{\textcolor[rgb]{0.25,0.50,0.50}{##1}}}
\expandafter\def\csname PY@tok@cp\endcsname{\def\PY@tc##1{\textcolor[rgb]{0.74,0.48,0.00}{##1}}}
\expandafter\def\csname PY@tok@k\endcsname{\let\PY@bf=\textbf\def\PY@tc##1{\textcolor[rgb]{0.00,0.50,0.00}{##1}}}
\expandafter\def\csname PY@tok@kp\endcsname{\def\PY@tc##1{\textcolor[rgb]{0.00,0.50,0.00}{##1}}}
\expandafter\def\csname PY@tok@kt\endcsname{\def\PY@tc##1{\textcolor[rgb]{0.69,0.00,0.25}{##1}}}
\expandafter\def\csname PY@tok@o\endcsname{\def\PY@tc##1{\textcolor[rgb]{0.40,0.40,0.40}{##1}}}
\expandafter\def\csname PY@tok@ow\endcsname{\let\PY@bf=\textbf\def\PY@tc##1{\textcolor[rgb]{0.67,0.13,1.00}{##1}}}
\expandafter\def\csname PY@tok@nb\endcsname{\def\PY@tc##1{\textcolor[rgb]{0.00,0.50,0.00}{##1}}}
\expandafter\def\csname PY@tok@nf\endcsname{\def\PY@tc##1{\textcolor[rgb]{0.00,0.00,1.00}{##1}}}
\expandafter\def\csname PY@tok@nc\endcsname{\let\PY@bf=\textbf\def\PY@tc##1{\textcolor[rgb]{0.00,0.00,1.00}{##1}}}
\expandafter\def\csname PY@tok@nn\endcsname{\let\PY@bf=\textbf\def\PY@tc##1{\textcolor[rgb]{0.00,0.00,1.00}{##1}}}
\expandafter\def\csname PY@tok@ne\endcsname{\let\PY@bf=\textbf\def\PY@tc##1{\textcolor[rgb]{0.82,0.25,0.23}{##1}}}
\expandafter\def\csname PY@tok@nv\endcsname{\def\PY@tc##1{\textcolor[rgb]{0.10,0.09,0.49}{##1}}}
\expandafter\def\csname PY@tok@no\endcsname{\def\PY@tc##1{\textcolor[rgb]{0.53,0.00,0.00}{##1}}}
\expandafter\def\csname PY@tok@nl\endcsname{\def\PY@tc##1{\textcolor[rgb]{0.63,0.63,0.00}{##1}}}
\expandafter\def\csname PY@tok@ni\endcsname{\let\PY@bf=\textbf\def\PY@tc##1{\textcolor[rgb]{0.60,0.60,0.60}{##1}}}
\expandafter\def\csname PY@tok@na\endcsname{\def\PY@tc##1{\textcolor[rgb]{0.49,0.56,0.16}{##1}}}
\expandafter\def\csname PY@tok@nt\endcsname{\let\PY@bf=\textbf\def\PY@tc##1{\textcolor[rgb]{0.00,0.50,0.00}{##1}}}
\expandafter\def\csname PY@tok@nd\endcsname{\def\PY@tc##1{\textcolor[rgb]{0.67,0.13,1.00}{##1}}}
\expandafter\def\csname PY@tok@s\endcsname{\def\PY@tc##1{\textcolor[rgb]{0.73,0.13,0.13}{##1}}}
\expandafter\def\csname PY@tok@sd\endcsname{\let\PY@it=\textit\def\PY@tc##1{\textcolor[rgb]{0.73,0.13,0.13}{##1}}}
\expandafter\def\csname PY@tok@si\endcsname{\let\PY@bf=\textbf\def\PY@tc##1{\textcolor[rgb]{0.73,0.40,0.53}{##1}}}
\expandafter\def\csname PY@tok@se\endcsname{\let\PY@bf=\textbf\def\PY@tc##1{\textcolor[rgb]{0.73,0.40,0.13}{##1}}}
\expandafter\def\csname PY@tok@sr\endcsname{\def\PY@tc##1{\textcolor[rgb]{0.73,0.40,0.53}{##1}}}
\expandafter\def\csname PY@tok@ss\endcsname{\def\PY@tc##1{\textcolor[rgb]{0.10,0.09,0.49}{##1}}}
\expandafter\def\csname PY@tok@sx\endcsname{\def\PY@tc##1{\textcolor[rgb]{0.00,0.50,0.00}{##1}}}
\expandafter\def\csname PY@tok@m\endcsname{\def\PY@tc##1{\textcolor[rgb]{0.40,0.40,0.40}{##1}}}
\expandafter\def\csname PY@tok@gh\endcsname{\let\PY@bf=\textbf\def\PY@tc##1{\textcolor[rgb]{0.00,0.00,0.50}{##1}}}
\expandafter\def\csname PY@tok@gu\endcsname{\let\PY@bf=\textbf\def\PY@tc##1{\textcolor[rgb]{0.50,0.00,0.50}{##1}}}
\expandafter\def\csname PY@tok@gd\endcsname{\def\PY@tc##1{\textcolor[rgb]{0.63,0.00,0.00}{##1}}}
\expandafter\def\csname PY@tok@gi\endcsname{\def\PY@tc##1{\textcolor[rgb]{0.00,0.63,0.00}{##1}}}
\expandafter\def\csname PY@tok@gr\endcsname{\def\PY@tc##1{\textcolor[rgb]{1.00,0.00,0.00}{##1}}}
\expandafter\def\csname PY@tok@ge\endcsname{\let\PY@it=\textit}
\expandafter\def\csname PY@tok@gs\endcsname{\let\PY@bf=\textbf}
\expandafter\def\csname PY@tok@gp\endcsname{\let\PY@bf=\textbf\def\PY@tc##1{\textcolor[rgb]{0.00,0.00,0.50}{##1}}}
\expandafter\def\csname PY@tok@go\endcsname{\def\PY@tc##1{\textcolor[rgb]{0.53,0.53,0.53}{##1}}}
\expandafter\def\csname PY@tok@gt\endcsname{\def\PY@tc##1{\textcolor[rgb]{0.00,0.27,0.87}{##1}}}
\expandafter\def\csname PY@tok@err\endcsname{\def\PY@bc##1{\setlength{\fboxsep}{0pt}\fcolorbox[rgb]{1.00,0.00,0.00}{1,1,1}{\strut ##1}}}
\expandafter\def\csname PY@tok@kc\endcsname{\let\PY@bf=\textbf\def\PY@tc##1{\textcolor[rgb]{0.00,0.50,0.00}{##1}}}
\expandafter\def\csname PY@tok@kd\endcsname{\let\PY@bf=\textbf\def\PY@tc##1{\textcolor[rgb]{0.00,0.50,0.00}{##1}}}
\expandafter\def\csname PY@tok@kn\endcsname{\let\PY@bf=\textbf\def\PY@tc##1{\textcolor[rgb]{0.00,0.50,0.00}{##1}}}
\expandafter\def\csname PY@tok@kr\endcsname{\let\PY@bf=\textbf\def\PY@tc##1{\textcolor[rgb]{0.00,0.50,0.00}{##1}}}
\expandafter\def\csname PY@tok@bp\endcsname{\def\PY@tc##1{\textcolor[rgb]{0.00,0.50,0.00}{##1}}}
\expandafter\def\csname PY@tok@fm\endcsname{\def\PY@tc##1{\textcolor[rgb]{0.00,0.00,1.00}{##1}}}
\expandafter\def\csname PY@tok@vc\endcsname{\def\PY@tc##1{\textcolor[rgb]{0.10,0.09,0.49}{##1}}}
\expandafter\def\csname PY@tok@vg\endcsname{\def\PY@tc##1{\textcolor[rgb]{0.10,0.09,0.49}{##1}}}
\expandafter\def\csname PY@tok@vi\endcsname{\def\PY@tc##1{\textcolor[rgb]{0.10,0.09,0.49}{##1}}}
\expandafter\def\csname PY@tok@vm\endcsname{\def\PY@tc##1{\textcolor[rgb]{0.10,0.09,0.49}{##1}}}
\expandafter\def\csname PY@tok@sa\endcsname{\def\PY@tc##1{\textcolor[rgb]{0.73,0.13,0.13}{##1}}}
\expandafter\def\csname PY@tok@sb\endcsname{\def\PY@tc##1{\textcolor[rgb]{0.73,0.13,0.13}{##1}}}
\expandafter\def\csname PY@tok@sc\endcsname{\def\PY@tc##1{\textcolor[rgb]{0.73,0.13,0.13}{##1}}}
\expandafter\def\csname PY@tok@dl\endcsname{\def\PY@tc##1{\textcolor[rgb]{0.73,0.13,0.13}{##1}}}
\expandafter\def\csname PY@tok@s2\endcsname{\def\PY@tc##1{\textcolor[rgb]{0.73,0.13,0.13}{##1}}}
\expandafter\def\csname PY@tok@sh\endcsname{\def\PY@tc##1{\textcolor[rgb]{0.73,0.13,0.13}{##1}}}
\expandafter\def\csname PY@tok@s1\endcsname{\def\PY@tc##1{\textcolor[rgb]{0.73,0.13,0.13}{##1}}}
\expandafter\def\csname PY@tok@mb\endcsname{\def\PY@tc##1{\textcolor[rgb]{0.40,0.40,0.40}{##1}}}
\expandafter\def\csname PY@tok@mf\endcsname{\def\PY@tc##1{\textcolor[rgb]{0.40,0.40,0.40}{##1}}}
\expandafter\def\csname PY@tok@mh\endcsname{\def\PY@tc##1{\textcolor[rgb]{0.40,0.40,0.40}{##1}}}
\expandafter\def\csname PY@tok@mi\endcsname{\def\PY@tc##1{\textcolor[rgb]{0.40,0.40,0.40}{##1}}}
\expandafter\def\csname PY@tok@il\endcsname{\def\PY@tc##1{\textcolor[rgb]{0.40,0.40,0.40}{##1}}}
\expandafter\def\csname PY@tok@mo\endcsname{\def\PY@tc##1{\textcolor[rgb]{0.40,0.40,0.40}{##1}}}
\expandafter\def\csname PY@tok@ch\endcsname{\let\PY@it=\textit\def\PY@tc##1{\textcolor[rgb]{0.25,0.50,0.50}{##1}}}
\expandafter\def\csname PY@tok@cm\endcsname{\let\PY@it=\textit\def\PY@tc##1{\textcolor[rgb]{0.25,0.50,0.50}{##1}}}
\expandafter\def\csname PY@tok@cpf\endcsname{\let\PY@it=\textit\def\PY@tc##1{\textcolor[rgb]{0.25,0.50,0.50}{##1}}}
\expandafter\def\csname PY@tok@c1\endcsname{\let\PY@it=\textit\def\PY@tc##1{\textcolor[rgb]{0.25,0.50,0.50}{##1}}}
\expandafter\def\csname PY@tok@cs\endcsname{\let\PY@it=\textit\def\PY@tc##1{\textcolor[rgb]{0.25,0.50,0.50}{##1}}}

\def\PYZbs{\char`\\}
\def\PYZus{\char`\_}
\def\PYZob{\char`\{}
\def\PYZcb{\char`\}}
\def\PYZca{\char`\^}
\def\PYZam{\char`\&}
\def\PYZlt{\char`\<}
\def\PYZgt{\char`\>}
\def\PYZsh{\char`\#}
\def\PYZpc{\char`\%}
\def\PYZdl{\char`\$}
\def\PYZhy{\char`\-}
\def\PYZsq{\char`\'}
\def\PYZdq{\char`\"}
\def\PYZti{\char`\~}
% for compatibility with earlier versions
\def\PYZat{@}
\def\PYZlb{[}
\def\PYZrb{]}
\makeatother


    % Exact colors from NB
    \definecolor{incolor}{rgb}{0.0, 0.0, 0.5}
    \definecolor{outcolor}{rgb}{0.545, 0.0, 0.0}



    
    % Prevent overflowing lines due to hard-to-break entities
    \sloppy 
    % Setup hyperref package
    \hypersetup{
      breaklinks=true,  % so long urls are correctly broken across lines
      colorlinks=true,
      urlcolor=urlcolor,
      linkcolor=linkcolor,
      citecolor=citecolor,
      }
    % Slightly bigger margins than the latex defaults
    
    \geometry{verbose,tmargin=1in,bmargin=1in,lmargin=1in,rmargin=1in}
    
    

    \begin{document}
    
    
    \maketitle
    
    

    
    \textless{}img
src=``http://www.uoc.edu/portal/\_resources/common/imatges/marca\_UOC/UOC\_Masterbrand.jpg``,
align=''left``\textgreater{}

M2.951 - Tipologia i cicle de vida de les dades

2017-2 · Màster universitari en Ciència de dades (Data science)

Autors: Joan Bonnín i Jose L. Dolz

~

\hypertarget{pruxe0ctica-2-neteja-i-validaciuxf3-de-les-dades}{%
\section{Pràctica 2: Neteja i validació de les
dades}\label{pruxe0ctica-2-neteja-i-validaciuxf3-de-les-dades}}

    \hypertarget{descripciuxf3}{%
\subsection{Descripció}\label{descripciuxf3}}

L'objectiu d'aquesta activitat serà el tractament d'un \emph{dataset},
que pot ser el creat a la pràctica 1 o bé qualsevol dataset lliure
disponible a Kaggle (https://www.kaggle.com). Alguns exemples de
\emph{dataset} amb els que podeu treballar són:

Red Wine Quality
(https://www.kaggle.com/uciml/red-wine-quality-cortez-et-al-2009).

Titanic: Machine Learning from Disaster
(https://www.kaggle.com/c/titanic).

Predict Future Sales
(https://www.kaggle.com/c/competitive-data-science-predict-future-sales/).

Els últims dos exemples corresponen a competicions actives a Kaggle de
manera que, opcionalment, podríeu aprofitar el treball realitzat durant
la pràctica per entrar en alguna d'aquestes competicions. Seguint les
principals etapes d'un projecte analític, les diferents tasques a
realitzar (i justificar) són les següents:

    \begin{Verbatim}[commandchars=\\\{\}]
{\color{incolor}In [{\color{incolor}1}]:} \PY{c+c1}{\PYZsh{} ATENCIÓ: Es important tenir instal·lats al sistema els següents}
        \PY{c+c1}{\PYZsh{} paquets: pandas, numpy, scipy, matplotlib, sklearn, pydotplus, graphviz}
        \PY{c+c1}{\PYZsh{}}
        \PY{c+c1}{\PYZsh{} Si se està utilitzan Python 3, la forma d\PYZsq{}instal·lar un paquet és}
        \PY{c+c1}{\PYZsh{} executant la següent ordre des de la línia de comandes:}
        \PY{c+c1}{\PYZsh{} \PYZgt{} pyp3 install nom\PYZus{}paquet}
        
        \PY{c+c1}{\PYZsh{} Basic maths \PYZam{} data structures}
        \PY{k+kn}{import} \PY{n+nn}{pandas} \PY{k}{as} \PY{n+nn}{pd}
        \PY{k+kn}{import} \PY{n+nn}{numpy} \PY{k}{as} \PY{n+nn}{np}
        \PY{k+kn}{import} \PY{n+nn}{scipy}
        \PY{k+kn}{import} \PY{n+nn}{scipy}\PY{n+nn}{.}\PY{n+nn}{stats} \PY{k}{as} \PY{n+nn}{stats}
        \PY{k+kn}{import} \PY{n+nn}{itertools}
        \PY{k+kn}{import} \PY{n+nn}{math}
        
        \PY{c+c1}{\PYZsh{} Data rendering}
        \PY{k+kn}{from} \PY{n+nn}{IPython}\PY{n+nn}{.}\PY{n+nn}{display} \PY{k}{import} \PY{n}{display}
        \PY{k+kn}{from} \PY{n+nn}{IPython}\PY{n+nn}{.}\PY{n+nn}{display} \PY{k}{import} \PY{n}{Image}  
        \PY{k+kn}{import} \PY{n+nn}{matplotlib}\PY{n+nn}{.}\PY{n+nn}{pyplot} \PY{k}{as} \PY{n+nn}{plt}
        
        \PY{c+c1}{\PYZsh{} Random forest}
        \PY{k+kn}{from} \PY{n+nn}{sklearn}\PY{n+nn}{.}\PY{n+nn}{ensemble} \PY{k}{import} \PY{n}{RandomForestClassifier}
        \PY{k+kn}{from} \PY{n+nn}{sklearn} \PY{k}{import} \PY{n}{preprocessing}
        \PY{k+kn}{from} \PY{n+nn}{sklearn}\PY{n+nn}{.}\PY{n+nn}{model\PYZus{}selection} \PY{k}{import} \PY{n}{train\PYZus{}test\PYZus{}split}
        \PY{k+kn}{from} \PY{n+nn}{sklearn}\PY{n+nn}{.}\PY{n+nn}{metrics} \PY{k}{import} \PY{n}{confusion\PYZus{}matrix}
        
        \PY{c+c1}{\PYZsh{} Linear model}
        \PY{k+kn}{from} \PY{n+nn}{sklearn} \PY{k}{import} \PY{n}{linear\PYZus{}model}
        
        \PY{c+c1}{\PYZsh{} Render tree}
        \PY{k+kn}{from} \PY{n+nn}{sklearn}\PY{n+nn}{.}\PY{n+nn}{externals}\PY{n+nn}{.}\PY{n+nn}{six} \PY{k}{import} \PY{n}{StringIO}  
        \PY{k+kn}{from} \PY{n+nn}{sklearn}\PY{n+nn}{.}\PY{n+nn}{tree} \PY{k}{import} \PY{n}{export\PYZus{}graphviz}
        \PY{k+kn}{import} \PY{n+nn}{pydotplus}
        \PY{k+kn}{import} \PY{n+nn}{graphviz}
\end{Verbatim}


    \begin{Verbatim}[commandchars=\\\{\}]
{\color{incolor}In [{\color{incolor}2}]:} \PY{c+c1}{\PYZsh{} General constants}
        \PY{n}{PVAL\PYZus{}THRESHOLD} \PY{o}{=} \PY{l+m+mf}{0.05}
        \PY{n}{RANDOM\PYZus{}SEED} \PY{o}{=} \PY{l+m+mi}{2018}
        \PY{n}{np}\PY{o}{.}\PY{n}{random}\PY{o}{.}\PY{n}{seed}\PY{p}{(}\PY{n}{RANDOM\PYZus{}SEED}\PY{p}{)}
\end{Verbatim}


    \hypertarget{descripcio-del-dataset.-perque-es-important-i-quina-preguntaproblema-preten-respondre}{%
\subsection{1. Descripció del dataset. Perquè és important i quina
pregunta/problema pretèn
respondre?}\label{descripcio-del-dataset.-perque-es-important-i-quina-preguntaproblema-preten-respondre}}

     Per aquesta pràctica hem escollit el conjunt de dades de la qualitat
del vi negre que, com bé s'explica en la seva pàgina de Kaggle, tracta
de les variants del vi portuguès conegut com `Vinho Verde'. Els camps
que composen el dataset són tots numèrics i són els següents:

fixed acidity: acidesa fixa (g/l). Majoria dels àcids fixos o no
volàtils fàcilment.

volatile acidity: acidesa volàtil (g/l). Quantitat d'àcid acètic que en
gran quantitat porta al vi a tenir gust de vinagre.

citric acid: àcid cítric (g/l). Quantitat d'aquest àcid (normalment
petita) que pot donar sabor i frescor als vins

residual sugar: sucre residual (g/l). Quantitat de sucre que queda
després de la fermentació. És estrany trobar vins amb menys d'1 g/l i
els que tenen més de 45 g/l es consideren dolços.

chlorides: clorurs (g/l). Representa la quantitat de sal al vi.

free sulfur dioxide: diòxid de sofre lliure o SO2 (mg/l). Parts per
milió (ppm) del diòxid que queda lliure un cop es barreja en barrejar-se
amb el vi.

total sulfur dioxide: Diòxid de sofre total (mg/l). La suma (en ppm) de
la part lliure i la part fixada al vi de SO2.

density: densitat (g/l). Aquest valor dependrà de la quantitat d'alcohol
i sucre en el vi.

pH: descriu com d'àcid o bàsic és un vi en una escala de 0 (molt àcid) a
14 (molt bàsic); La majoria dels vins són entre 3 i 4 en l'escala de pH.

sulphates: sulfats (g/l), que contribueixen al SO2.

Aquest component prevé al vi de bacteris i de l'oxidació.

alcohol: percentatge de contingut alcohòlic en el vi. Volum d'etanol /
Volum del producte

quality: qualitat del vi en una puntuació entre el 0 i el 10.

Com podem veure, tenim els 12 atributs: 11 mesures físico-químiques i la
qualitat, que podríem dir que és la classe.

Aquest conjunt de dades és important perquè ens pot servir perquè,
mitjançant proves de correlació, podem veure quins atributs són més
influents a l'hora de millorar o empitjorar la qualitat d'un vi. A més,
entrenarem un model complex d'aprenentatge computacional que podrà
predir la qualitat d'un vi, mitjançant aquestes dades físico-químiques
obtingudes pels diferents instruments de mesurament.

    \hypertarget{integracio-i-seleccio-de-les-dades-dinteres-a-analitzar.}{%
\subsection{2. Integració i selecció de les dades d'interès a
analitzar.}\label{integracio-i-seleccio-de-les-dades-dinteres-a-analitzar.}}

    \begin{Verbatim}[commandchars=\\\{\}]
{\color{incolor}In [{\color{incolor}3}]:} \PY{n}{wine\PYZus{}df} \PY{o}{=} \PY{n}{pd}\PY{o}{.}\PY{n}{read\PYZus{}csv}\PY{p}{(}\PY{l+s+s1}{\PYZsq{}}\PY{l+s+s1}{winequality\PYZhy{}red.csv}\PY{l+s+s1}{\PYZsq{}}\PY{p}{)}
        \PY{n+nb}{print}\PY{p}{(}\PY{l+s+s2}{\PYZdq{}}\PY{l+s+s2}{El conjunt de dades presenta }\PY{l+s+si}{\PYZob{}\PYZcb{}}\PY{l+s+s2}{ camps i està compost per }\PY{l+s+si}{\PYZob{}\PYZcb{}}\PY{l+s+s2}{ registres.}\PY{l+s+s2}{\PYZdq{}}\PY{o}{.}\PY{n}{format}\PY{p}{(}\PY{n}{wine\PYZus{}df}\PY{o}{.}\PY{n}{shape}\PY{p}{[}\PY{l+m+mi}{1}\PY{p}{]}\PY{p}{,} \PY{n}{wine\PYZus{}df}\PY{o}{.}\PY{n}{shape}\PY{p}{[}\PY{l+m+mi}{0}\PY{p}{]}\PY{p}{)}\PY{p}{)}
        \PY{n}{display}\PY{p}{(}\PY{n}{wine\PYZus{}df}\PY{o}{.}\PY{n}{head}\PY{p}{(}\PY{p}{)}\PY{p}{)}
\end{Verbatim}


    \begin{Verbatim}[commandchars=\\\{\}]
El conjunt de dades presenta 12 camps i està compost per 1599 registres.

    \end{Verbatim}

    
    \begin{verbatim}
   fixed acidity  volatile acidity  citric acid  residual sugar  chlorides  \
0            7.4              0.70         0.00             1.9      0.076   
1            7.8              0.88         0.00             2.6      0.098   
2            7.8              0.76         0.04             2.3      0.092   
3           11.2              0.28         0.56             1.9      0.075   
4            7.4              0.70         0.00             1.9      0.076   

   free sulfur dioxide  total sulfur dioxide  density    pH  sulphates  \
0                 11.0                  34.0   0.9978  3.51       0.56   
1                 25.0                  67.0   0.9968  3.20       0.68   
2                 15.0                  54.0   0.9970  3.26       0.65   
3                 17.0                  60.0   0.9980  3.16       0.58   
4                 11.0                  34.0   0.9978  3.51       0.56   

   alcohol  quality  
0      9.4        5  
1      9.8        5  
2      9.8        5  
3      9.8        6  
4      9.4        5  
    \end{verbatim}

    
    Com podem observar en la taula, tots els atributs presenten valors
numèrics continus que ens poden servir per cercar una relació lineal amb
el valor discret de la qualitat. En aquest punt, no podem prescindir de
cap d'aquests atributs doncs no sabem quina és la seva relació amb la
nota de qualitat. Per tant, no ens desfarem de cap.

Per altra banda, crearem un atribut categòric booleà que etiqueti si un
vi és bo o dolent, depenent de la seva nota. En el nostre cas, escollim
que els bons vins són aquells que tenen una puntuació de qualitat de 6 o
superior. Aquest atribut ens servirà com a classificador per la creació
d'un model basat en \emph{Random Forests}.

\begin{verbatim}
</font>
    
\end{verbatim}

    \hypertarget{neteja-de-les-dades.}{%
\subsection{3. Neteja de les dades.}\label{neteja-de-les-dades.}}

    \hypertarget{les-dades-contenen-zeros-o-elements-buits-com-gestionaries-aquests-casos}{%
\subsubsection{3.1. Les dades contenen zeros o elements buits? Com
gestionaries aquests
casos?}\label{les-dades-contenen-zeros-o-elements-buits-com-gestionaries-aquests-casos}}

    \begin{Verbatim}[commandchars=\\\{\}]
{\color{incolor}In [{\color{incolor}4}]:} \PY{c+c1}{\PYZsh{} Contem els valors nuls}
        \PY{n}{n\PYZus{}nulls} \PY{o}{=} \PY{n}{wine\PYZus{}df}\PY{p}{[}\PY{n}{wine\PYZus{}df}\PY{o}{.}\PY{n}{isnull}\PY{p}{(}\PY{p}{)}\PY{p}{]}\PY{o}{.}\PY{n}{count}\PY{p}{(}\PY{p}{)}
        
        \PY{c+c1}{\PYZsh{} Contem els zeros}
        \PY{n}{n\PYZus{}zeros} \PY{o}{=} \PY{n}{wine\PYZus{}df}\PY{p}{[}\PY{n}{wine\PYZus{}df} \PY{o}{==} \PY{l+m+mi}{0}\PY{p}{]}\PY{o}{.}\PY{n}{count}\PY{p}{(}\PY{p}{)}
        
        \PY{n}{df\PYZus{}empties} \PY{o}{=} \PY{n}{pd}\PY{o}{.}\PY{n}{DataFrame}\PY{p}{(}\PY{p}{)}
        \PY{n}{df\PYZus{}empties}\PY{p}{[}\PY{l+s+s2}{\PYZdq{}}\PY{l+s+s2}{Nombre de Nulls}\PY{l+s+s2}{\PYZdq{}}\PY{p}{]} \PY{o}{=} \PY{n}{n\PYZus{}nulls}
        \PY{n}{df\PYZus{}empties}\PY{p}{[}\PY{l+s+s2}{\PYZdq{}}\PY{l+s+s2}{Nombre de zeros}\PY{l+s+s2}{\PYZdq{}}\PY{p}{]} \PY{o}{=} \PY{n}{n\PYZus{}zeros}
        \PY{n}{display}\PY{p}{(}\PY{n}{df\PYZus{}empties}\PY{o}{.}\PY{n}{T}\PY{p}{)}
\end{Verbatim}


    
    \begin{verbatim}
                 fixed acidity  volatile acidity  citric acid  residual sugar  \
Nombre de Nulls              0                 0            0               0   
Nombre de zeros              0                 0          132               0   

                 chlorides  free sulfur dioxide  total sulfur dioxide  \
Nombre de Nulls          0                    0                     0   
Nombre de zeros          0                    0                     0   

                 density  pH  sulphates  alcohol  quality  
Nombre de Nulls        0   0          0        0        0  
Nombre de zeros        0   0          0        0        0  
    \end{verbatim}

    
    Com podem observar, no tenim cap atribut que presenti valors nuls. Per
altra banda, només trobem zeros en l'atribut d'àcid cítric.
Concretament, tenim 132 zeros d'un total de 1599 registres, el que
representa un 8'26\% del total.

En aquest cas, no cal que substituïm els zeros per cap altre valor. És
totalment normal trobar vins negres sense àcid cítric, fet que els hi
dóna un sabor més anyenc. De fet, com veurem més endavant, el rang de
grams per litre de l'àcid cítric en vins negres és molt petit i el zero
estaria dintre dels barems. Fins i tot podem trobar vins blancs amb zero
grams d'àcid cítric, tot i que en aquests acostuma a haver-hi unes
quantitats més altes que als vins negres.

Així doncs, podem observar que en aquest sentit, pel que fa a zeros i
valors nuls, el \emph{dataset} està completament net des de l'inici. És
bastant raonable pensar que Kaggle ha realitzat una tasca prèvia de
neteja per facilitar l'ús de les dades amb intencions analítiques, sense
haver de dedicar gaire esforç a la neteja inicial.

    \hypertarget{identificacio-i-tractament-de-valors-extrems.}{%
\subsubsection{3.2. Identificació i tractament de valors
extrems.}\label{identificacio-i-tractament-de-valors-extrems.}}

    \begin{Verbatim}[commandchars=\\\{\}]
{\color{incolor}In [{\color{incolor}5}]:} \PY{c+c1}{\PYZsh{} Dibueixem un boxplot per cada atribut per veure màxims, }
        \PY{c+c1}{\PYZsh{} mínims, quartils, rang interquartilic i outliers.}
        
        \PY{n}{df\PYZus{}cols} \PY{o}{=} \PY{n+nb}{list}\PY{p}{(}\PY{n}{wine\PYZus{}df}\PY{p}{)}
        \PY{n}{n\PYZus{}rows} \PY{o}{=} \PY{l+m+mi}{2}
        \PY{n}{n\PYZus{}cols} \PY{o}{=} \PY{l+m+mi}{6}
        
        \PY{n}{fig}\PY{p}{,} \PY{n}{axes} \PY{o}{=} \PY{n}{plt}\PY{o}{.}\PY{n}{subplots}\PY{p}{(}\PY{n}{n\PYZus{}rows}\PY{p}{,} \PY{n}{n\PYZus{}cols}\PY{p}{,} \PY{n}{figsize}\PY{o}{=}\PY{p}{(}\PY{l+m+mi}{30}\PY{p}{,} \PY{l+m+mi}{15}\PY{p}{)}\PY{p}{)}
        \PY{k}{for} \PY{n}{i}\PY{p}{,} \PY{n}{row\PYZus{}axes} \PY{o+ow}{in} \PY{n+nb}{enumerate}\PY{p}{(}\PY{n}{axes}\PY{p}{)}\PY{p}{:}
            \PY{k}{for} \PY{n}{j}\PY{p}{,} \PY{n}{ax} \PY{o+ow}{in} \PY{n+nb}{enumerate}\PY{p}{(}\PY{n}{row\PYZus{}axes}\PY{p}{)}\PY{p}{:}
                \PY{n}{idx} \PY{o}{=} \PY{n}{i}\PY{o}{*}\PY{n}{n\PYZus{}cols} \PY{o}{+} \PY{n}{j}
                \PY{n}{wine\PYZus{}df}\PY{o}{.}\PY{n}{boxplot}\PY{p}{(}\PY{n}{column}\PY{o}{=}\PY{n}{df\PYZus{}cols}\PY{p}{[}\PY{n}{idx}\PY{p}{]}\PY{p}{,} \PY{n}{ax}\PY{o}{=}\PY{n}{axes}\PY{p}{[}\PY{n}{i}\PY{p}{]}\PY{p}{[}\PY{n}{j}\PY{p}{]}\PY{p}{,} \PY{n}{grid}\PY{o}{=}\PY{k+kc}{False}\PY{p}{)}
                
        \PY{n}{plt}\PY{o}{.}\PY{n}{show}\PY{p}{(}\PY{p}{)}
\end{Verbatim}


    \begin{center}
    \adjustimage{max size={0.9\linewidth}{0.9\paperheight}}{output_14_0.png}
    \end{center}
    { \hspace*{\fill} \\}
    
    Als \emph{boxplots} s'observa que la majoria de variables presenten una
quantitat elevada d'\emph{outliers}, així que val la pena aplicar-hi una
anàlisi més extensa per detectar si es tracta de presència de valors
erronis o la realitat de les dades és la que es mostra.

El fet que la majoria d'\emph{outliers} s'agrupin a prop dels bigotis i
la densitat baixi segons s'allunyen, ens fa pensar que els valors poden
ser correctes, i que simplement la distribució presenta una desviació
lateral o una variància molt elevada. Un exemple habitual per mostrar
dades que presenten aquesta característica són els salaris.

    \hypertarget{analisi-de-les-dades.}{%
\subsection{4. Anàlisi de les dades.}\label{analisi-de-les-dades.}}

\#\#\#~4.1. Selecció dels grups de dades que es volen analitzar/comparar
(planificació dels anàlisis a aplicar).

    No descartarem cap dels atributs que ofereix el conjunt de dades
proporcionat perquè volem esbrinar quins són els que tenen més
importància, quins ofereixen una distribució normal i quins no, com és
la relació de variàncies, l'anàlisi de correlacions i, finalment, un
model de classificació.

\begin{verbatim}
</font>
\end{verbatim}

    \hypertarget{comprovacio-de-la-normalitat-i-homogeneitat-de-la-variancia.}{%
\subsubsection{4.2. Comprovació de la normalitat i homogeneïtat de la
variància.}\label{comprovacio-de-la-normalitat-i-homogeneitat-de-la-variancia.}}

    El Teorema del Límit Central ens diu que qualsevol població amb un
nombre d'elements prou gran tendeix a una distribució normal estàndard.
En el nostre cas, considerem que aquest Teorema es pot aplicar quan N
\textgreater{} 30. Per tant, el nostre conjunt tendeix a una distribució
normal.

\begin{verbatim}
<p>
    De totes maneres, a continuació fem servir histogrames i el test de normalitat d'Anderson-Darling per assegurar-nos:
</p>
\end{verbatim}

    \begin{Verbatim}[commandchars=\\\{\}]
{\color{incolor}In [{\color{incolor}6}]:} \PY{c+c1}{\PYZsh{}\PYZsh{}\PYZsh{}\PYZsh{}\PYZsh{}\PYZsh{}\PYZsh{} NORMALITAT \PYZsh{}\PYZsh{}\PYZsh{}\PYZsh{}\PYZsh{}\PYZsh{}\PYZsh{}}
        
        \PY{c+c1}{\PYZsh{} Mètodes gràfics }
        \PY{k}{def} \PY{n+nf}{render\PYZus{}normality\PYZus{}histograms}\PY{p}{(}\PY{p}{)}\PY{p}{:}
            \PY{n}{n\PYZus{}rows} \PY{o}{=} \PY{l+m+mi}{3}
            \PY{n}{n\PYZus{}cols} \PY{o}{=} \PY{l+m+mi}{4}
            \PY{n}{n\PYZus{}bars} \PY{o}{=} \PY{l+m+mi}{30}
            \PY{n}{fig}\PY{p}{,} \PY{n}{axes} \PY{o}{=} \PY{n}{plt}\PY{o}{.}\PY{n}{subplots}\PY{p}{(}\PY{n}{n\PYZus{}rows}\PY{p}{,} \PY{n}{n\PYZus{}cols}\PY{p}{,} \PY{n}{figsize}\PY{o}{=}\PY{p}{(}\PY{l+m+mi}{30}\PY{p}{,} \PY{l+m+mi}{15}\PY{p}{)}\PY{p}{)}
            \PY{k}{for} \PY{n}{i}\PY{p}{,} \PY{n}{row\PYZus{}axes} \PY{o+ow}{in} \PY{n+nb}{enumerate}\PY{p}{(}\PY{n}{axes}\PY{p}{)}\PY{p}{:}
                \PY{k}{for} \PY{n}{j}\PY{p}{,} \PY{n}{ax} \PY{o+ow}{in} \PY{n+nb}{enumerate}\PY{p}{(}\PY{n}{row\PYZus{}axes}\PY{p}{)}\PY{p}{:}
                    \PY{n}{idx} \PY{o}{=} \PY{n}{i}\PY{o}{*}\PY{n}{n\PYZus{}cols} \PY{o}{+} \PY{n}{j}
        
                    \PY{n}{attr} \PY{o}{=} \PY{n}{df\PYZus{}cols}\PY{p}{[}\PY{n}{idx}\PY{p}{]}
                    \PY{n}{ax}\PY{o}{.}\PY{n}{set\PYZus{}xlabel}\PY{p}{(}\PY{n}{attr}\PY{p}{)}      
        
                    \PY{n}{data} \PY{o}{=} \PY{n}{wine\PYZus{}df}\PY{p}{[}\PY{n}{attr}\PY{p}{]}\PY{o}{.}\PY{n}{sort\PYZus{}values}\PY{p}{(}\PY{p}{)}
                    \PY{n}{ax}\PY{o}{.}\PY{n}{hist}\PY{p}{(}\PY{n}{data}\PY{p}{,} \PY{n}{bins}\PY{o}{=}\PY{n}{n\PYZus{}bars}\PY{p}{,} \PY{n}{density}\PY{o}{=}\PY{k+kc}{True}\PY{p}{)}
        
                    \PY{n}{norm} \PY{o}{=} \PY{n}{stats}\PY{o}{.}\PY{n}{norm}\PY{o}{.}\PY{n}{pdf}\PY{p}{(}\PY{n}{data}\PY{p}{,} \PY{n}{np}\PY{o}{.}\PY{n}{mean}\PY{p}{(}\PY{n}{data}\PY{p}{)}\PY{p}{,} \PY{n}{np}\PY{o}{.}\PY{n}{std}\PY{p}{(}\PY{n}{data}\PY{p}{)}\PY{p}{)}
                    \PY{n}{ax}\PY{o}{.}\PY{n}{plot}\PY{p}{(}\PY{n}{data}\PY{p}{,} \PY{n}{norm}\PY{p}{)} 
            \PY{n}{plt}\PY{o}{.}\PY{n}{show}\PY{p}{(}\PY{p}{)}
            
        \PY{k}{def} \PY{n+nf}{render\PYZus{}normality\PYZus{}table}\PY{p}{(}\PY{p}{)}\PY{p}{:}
            \PY{p}{(}\PY{n}{anderson\PYZus{}true}\PY{p}{,} \PY{n}{anderson\PYZus{}false}\PY{p}{)} \PY{o}{=} \PY{n}{test\PYZus{}normality\PYZus{}anderson}\PY{p}{(}\PY{p}{)}
               
            \PY{n}{anderson} \PY{o}{=} \PY{n}{merge\PYZus{}normality\PYZus{}results}\PY{p}{(}\PY{n}{anderson\PYZus{}true}\PY{p}{,} \PY{n}{anderson\PYZus{}false}\PY{p}{)}
            
            \PY{n}{normality\PYZus{}res} \PY{o}{=} \PY{n}{pd}\PY{o}{.}\PY{n}{DataFrame}\PY{p}{(}\PY{p}{[}\PY{n}{anderson}\PY{p}{]}\PY{p}{,}
                                         \PY{n}{index}\PY{o}{=}\PY{p}{[}\PY{l+s+s2}{\PYZdq{}}\PY{l+s+s2}{Test Anderson\PYZhy{}Darling}\PY{l+s+s2}{\PYZdq{}}\PY{p}{]}\PY{p}{)}
            \PY{n}{normality\PYZus{}res}\PY{o}{.}\PY{n}{columns}\PY{o}{.}\PY{n}{name} \PY{o}{=} \PY{l+s+s2}{\PYZdq{}}\PY{l+s+s2}{Segueix distribució normal?}\PY{l+s+s2}{\PYZdq{}}
            \PY{n}{display}\PY{p}{(}\PY{n}{normality\PYZus{}res}\PY{p}{)}
                
                
        \PY{c+c1}{\PYZsh{} Mètodes numèrics}
        \PY{k}{def} \PY{n+nf}{test\PYZus{}normality\PYZus{}anderson}\PY{p}{(}\PY{n}{p\PYZus{}value\PYZus{}threshold}\PY{o}{=}\PY{n}{PVAL\PYZus{}THRESHOLD}\PY{p}{,} \PY{n}{render}\PY{o}{=}\PY{k+kc}{False}\PY{p}{)}\PY{p}{:}
            \PY{n}{normal\PYZus{}attrs} \PY{o}{=} \PY{p}{[}\PY{p}{]}
            \PY{n}{non\PYZus{}normal\PYZus{}attrs} \PY{o}{=} \PY{p}{[}\PY{p}{]}
            \PY{k}{for} \PY{n}{attr\PYZus{}key} \PY{o+ow}{in} \PY{n}{wine\PYZus{}df}\PY{p}{:}
                \PY{n}{attr} \PY{o}{=} \PY{n}{wine\PYZus{}df}\PY{p}{[}\PY{n}{attr\PYZus{}key}\PY{p}{]}\PY{o}{.}\PY{n}{values}
                \PY{n}{res} \PY{o}{=} \PY{n}{stats}\PY{o}{.}\PY{n}{anderson}\PY{p}{(}\PY{n}{attr}\PY{p}{,} \PY{n}{dist}\PY{o}{=}\PY{l+s+s1}{\PYZsq{}}\PY{l+s+s1}{norm}\PY{l+s+s1}{\PYZsq{}}\PY{p}{)}
                \PY{n}{stat} \PY{o}{=} \PY{n}{res}\PY{o}{.}\PY{n}{statistic}
                \PY{n}{threshold} \PY{o}{=} \PY{n}{res}\PY{o}{.}\PY{n}{critical\PYZus{}values}\PY{p}{[}\PY{l+m+mi}{2}\PY{p}{]} \PY{c+c1}{\PYZsh{}0.05 significance == PVAL\PYZus{}THRESHOLD}
                \PY{k}{if} \PY{n}{stat} \PY{o}{\PYZgt{}} \PY{n}{threshold}\PY{p}{:}
                    \PY{n}{normal\PYZus{}attrs}\PY{o}{.}\PY{n}{append}\PY{p}{(}\PY{n}{attr\PYZus{}key}\PY{p}{)}
                \PY{k}{else}\PY{p}{:}
                    \PY{n}{non\PYZus{}normal\PYZus{}attrs}\PY{o}{.}\PY{n}{append}\PY{p}{(}\PY{n}{attr\PYZus{}key}\PY{p}{)}
                    
            \PY{k}{if} \PY{n}{render}\PY{p}{:}
                \PY{n}{print\PYZus{}attrs\PYZus{}distributions}\PY{p}{(}\PY{n}{normal\PYZus{}attrs}\PY{p}{,} \PY{n}{non\PYZus{}normal\PYZus{}attrs}\PY{p}{)}
                
            \PY{k}{return} \PY{p}{(}\PY{n}{normal\PYZus{}attrs}\PY{p}{,} \PY{n}{non\PYZus{}normal\PYZus{}attrs}\PY{p}{)}
        
            
        \PY{k}{def} \PY{n+nf}{print\PYZus{}attrs\PYZus{}distributions}\PY{p}{(}\PY{n}{normal}\PY{p}{,} \PY{n}{non\PYZus{}normal}\PY{p}{)}\PY{p}{:}
            \PY{n+nb}{print}\PY{p}{(}\PY{l+s+s2}{\PYZdq{}}\PY{l+s+s2}{Atributs amb distribució normal: }\PY{l+s+si}{\PYZob{}\PYZcb{}}\PY{l+s+s2}{\PYZdq{}}\PY{o}{.}\PY{n}{format}\PY{p}{(}\PY{n}{normal}\PY{p}{)}\PY{p}{)}
            \PY{n+nb}{print}\PY{p}{(}\PY{l+s+s2}{\PYZdq{}}\PY{l+s+s2}{Atributs amb distribució no normal: }\PY{l+s+si}{\PYZob{}\PYZcb{}}\PY{l+s+s2}{\PYZdq{}}\PY{o}{.}\PY{n}{format}\PY{p}{(}\PY{n}{non\PYZus{}normal}\PY{p}{)}\PY{p}{)}
        
            
        \PY{k}{def} \PY{n+nf}{merge\PYZus{}normality\PYZus{}results}\PY{p}{(}\PY{n}{normal}\PY{p}{,} \PY{n}{non\PYZus{}normal}\PY{p}{)}\PY{p}{:}
            \PY{n}{dict\PYZus{}normal} \PY{o}{=} \PY{p}{\PYZob{}}\PY{n}{k}\PY{p}{:} \PY{l+s+s1}{\PYZsq{}}\PY{l+s+s1}{✓}\PY{l+s+s1}{\PYZsq{}} \PY{k}{for} \PY{n}{k} \PY{o+ow}{in} \PY{n}{normal}\PY{p}{\PYZcb{}}
            \PY{n}{dict\PYZus{}non\PYZus{}normal} \PY{o}{=} \PY{p}{\PYZob{}}\PY{n}{k}\PY{p}{:} \PY{l+s+s1}{\PYZsq{}}\PY{l+s+s1}{✗}\PY{l+s+s1}{\PYZsq{}} \PY{k}{for} \PY{n}{k} \PY{o+ow}{in} \PY{n}{non\PYZus{}normal}\PY{p}{\PYZcb{}}
            
            \PY{c+c1}{\PYZsh{}inplace\PYZus{}merging}
            \PY{n}{dict\PYZus{}normal}\PY{o}{.}\PY{n}{update}\PY{p}{(}\PY{n}{dict\PYZus{}non\PYZus{}normal}\PY{p}{)}    
            \PY{k}{return} \PY{n}{dict\PYZus{}normal}
        
        \PY{n}{render\PYZus{}normality\PYZus{}histograms}\PY{p}{(}\PY{p}{)}
        \PY{n}{render\PYZus{}normality\PYZus{}table}\PY{p}{(}\PY{p}{)}
\end{Verbatim}


    \begin{center}
    \adjustimage{max size={0.9\linewidth}{0.9\paperheight}}{output_20_0.png}
    \end{center}
    { \hspace*{\fill} \\}
    
    
    \begin{verbatim}
Segueix distribució normal? alcohol chlorides citric acid density  \
Test Anderson-Darling             ✓         ✓           ✓       ✓   

Segueix distribució normal? fixed acidity free sulfur dioxide pH quality  \
Test Anderson-Darling                   ✓                   ✓  ✓       ✓   

Segueix distribució normal? residual sugar sulphates total sulfur dioxide  \
Test Anderson-Darling                    ✓         ✓                    ✓   

Segueix distribució normal? volatile acidity  
Test Anderson-Darling                      ✓  
    \end{verbatim}

    
    El test d'Anderson-Darling ens mostra com tots els atributs segueixen
una distribució normal. Això també ho corrobora els histogrames i la
corba de distribució normal. Tot i això, amb grans conjunts de dades,
els tests numèrics sobre normalitat no són gaire fiables. Per tant, tal
com veiem a les corbes i el que hem comentat sobre el Teorema del Límit
Central, considerarem que tots els atributs segueixen una distribució
normal.

Per avaluar l'homogeneitat de la variància disposem de diferents
tècniques i eines. Entre les més conegudes trobem \emph{Levene} i
\emph{Fligner}, així que passem a aplicar-los mitjançant la
implementació d'Scipy.

\begin{verbatim}
</font>
\end{verbatim}

    \begin{Verbatim}[commandchars=\\\{\}]
{\color{incolor}In [{\color{incolor}7}]:} \PY{c+c1}{\PYZsh{}\PYZsh{}\PYZsh{}\PYZsh{}\PYZsh{}\PYZsh{}\PYZsh{} HOMOGENEITAT VARIANCIA \PYZsh{}\PYZsh{}\PYZsh{}\PYZsh{}\PYZsh{}\PYZsh{}\PYZsh{}}
        
        \PY{c+c1}{\PYZsh{} Levene és util si la normalitat no està asegurada}
        \PY{k}{def} \PY{n+nf}{levene\PYZus{}test}\PY{p}{(}\PY{n}{threshold}\PY{o}{=}\PY{n}{PVAL\PYZus{}THRESHOLD}\PY{p}{)}\PY{p}{:}
            \PY{n}{res} \PY{o}{=} \PY{n}{stats}\PY{o}{.}\PY{n}{levene}\PY{p}{(}\PY{o}{*}\PY{n}{wine\PYZus{}df}\PY{o}{.}\PY{n}{as\PYZus{}matrix}\PY{p}{(}\PY{p}{)}\PY{p}{,} \PY{n}{center}\PY{o}{=}\PY{l+s+s1}{\PYZsq{}}\PY{l+s+s1}{median}\PY{l+s+s1}{\PYZsq{}}\PY{p}{)} \PY{c+c1}{\PYZsh{} Expand matrix to n parameters with \PYZsq{}*\PYZsq{}}
            \PY{k}{return} \PY{n}{res}\PY{o}{.}\PY{n}{pvalue}
        \PY{c+c1}{\PYZsh{}     return res}
            \PY{n}{p} \PY{o}{=} \PY{n}{res}\PY{o}{.}\PY{n}{pvalue}
            \PY{n}{pprint}\PY{p}{(}\PY{p}{(}\PY{l+s+s2}{\PYZdq{}}\PY{l+s+s2}{Levene}\PY{l+s+s2}{\PYZdq{}}\PY{p}{,} \PY{n}{p}\PY{p}{)}\PY{p}{)}
            \PY{k}{return} \PY{n}{p} \PY{o}{\PYZgt{}} \PY{n}{threshold}
        
        \PY{k}{def} \PY{n+nf}{fligner\PYZus{}test}\PY{p}{(}\PY{n}{threshold}\PY{o}{=}\PY{n}{PVAL\PYZus{}THRESHOLD}\PY{p}{)}\PY{p}{:}
            \PY{n}{res} \PY{o}{=} \PY{n}{stats}\PY{o}{.}\PY{n}{fligner}\PY{p}{(}\PY{o}{*}\PY{n}{wine\PYZus{}df}\PY{o}{.}\PY{n}{as\PYZus{}matrix}\PY{p}{(}\PY{p}{)}\PY{p}{,} \PY{n}{center}\PY{o}{=}\PY{l+s+s1}{\PYZsq{}}\PY{l+s+s1}{median}\PY{l+s+s1}{\PYZsq{}}\PY{p}{)} \PY{c+c1}{\PYZsh{} Expand matrix to n parameters with \PYZsq{}*\PYZsq{}}
            \PY{k}{return} \PY{n}{res}\PY{o}{.}\PY{n}{pvalue}
        \PY{c+c1}{\PYZsh{}     return res}
            \PY{n}{p} \PY{o}{=} \PY{n}{res}\PY{o}{.}\PY{n}{pvalue}
            \PY{n}{pprint}\PY{p}{(}\PY{p}{(}\PY{l+s+s2}{\PYZdq{}}\PY{l+s+s2}{Flinger}\PY{l+s+s2}{\PYZdq{}}\PY{p}{,} \PY{n}{p}\PY{p}{)}\PY{p}{)}
            \PY{k}{return} \PY{n}{p} \PY{o}{\PYZgt{}} \PY{n}{threshold}
        
        
        \PY{n}{variance\PYZus{}tests} \PY{o}{=} \PY{p}{[}\PY{n}{levene\PYZus{}test}\PY{p}{(}\PY{p}{)}\PY{p}{,} \PY{n}{fligner\PYZus{}test}\PY{p}{(}\PY{p}{)}\PY{p}{]}
        \PY{n}{are\PYZus{}equal} \PY{o}{=} \PY{p}{[}\PY{n}{p} \PY{o}{\PYZgt{}} \PY{n}{PVAL\PYZus{}THRESHOLD} \PY{k}{for} \PY{n}{p} \PY{o+ow}{in} \PY{n}{variance\PYZus{}tests}\PY{p}{]}
        
        \PY{n}{display}\PY{p}{(}\PY{n}{pd}\PY{o}{.}\PY{n}{DataFrame}\PY{p}{(}
            \PY{p}{\PYZob{}}
                \PY{l+s+s1}{\PYZsq{}}\PY{l+s+s1}{Populations with equal variances?}\PY{l+s+s1}{\PYZsq{}}\PY{p}{:} \PY{n}{are\PYZus{}equal}\PY{p}{,}
                \PY{l+s+s1}{\PYZsq{}}\PY{l+s+s1}{p\PYZhy{}value}\PY{l+s+s1}{\PYZsq{}}\PY{p}{:} \PY{n}{variance\PYZus{}tests}
            \PY{p}{\PYZcb{}}\PY{p}{,}
            \PY{n}{index}\PY{o}{=}\PY{p}{[}\PY{l+s+s1}{\PYZsq{}}\PY{l+s+s1}{Levene test}\PY{l+s+s1}{\PYZsq{}}\PY{p}{,} \PY{l+s+s1}{\PYZsq{}}\PY{l+s+s1}{Fligner test}\PY{l+s+s1}{\PYZsq{}}\PY{p}{]}
        \PY{p}{)}\PY{o}{.}\PY{n}{T}\PY{p}{)}
\end{Verbatim}


    
    \begin{verbatim}
                                  Levene test Fligner test
Populations with equal variances?        True         True
p-value                                     1            1
    \end{verbatim}

    
    Com podem observar, les poblacions passen el test de Levene i el test de
Fligner i, per tant podem assegurar que existeix una homogeneïtat en les
variàncies dels atributs.

\begin{verbatim}
</font>
\end{verbatim}

    \hypertarget{aplicacio-de-proves-estadistiques-per-comparar-els-grups-de-dades.-en-funcio-de-les-dades-i-de-lobjectiu-de-lestudi-aplicar-proves-de-contrast-dhipotesis-correlacions-regressions-etc.}{%
\subsubsection{4.3. Aplicació de proves estadístiques per comparar els
grups de dades. En funció de les dades i de l'objectiu de l'estudi,
aplicar proves de contrast d'hipòtesis, correlacions, regressions,
etc.}\label{aplicacio-de-proves-estadistiques-per-comparar-els-grups-de-dades.-en-funcio-de-les-dades-i-de-lobjectiu-de-lestudi-aplicar-proves-de-contrast-dhipotesis-correlacions-regressions-etc.}}

    Per aplicar diferents estudis sobre les dades, hem definit dues
preguntes a respondre: \emph{Quines característiques determinen la
qualitat del vi?} i \emph{Es pot inferir, a partir de la característica
més important, la qualitat del vi?} 

    \hypertarget{quines-caracteruxedstiques-determinen-la-qualitat-del-vi}{%
\subsubsection{Quines característiques determinen la qualitat del
vi?}\label{quines-caracteruxedstiques-determinen-la-qualitat-del-vi}}

La primera aproximació per determinar quines variables tenen més pes
sobre la qualitat final és cercar si existeix una correlació entre els
atributs i la qualitat i quin és el pes de cadascun.

    \begin{Verbatim}[commandchars=\\\{\}]
{\color{incolor}In [{\color{incolor}8}]:} \PY{c+c1}{\PYZsh{} Mirem les correlacions entre els atributs i la qualitat}
        \PY{n}{data} \PY{o}{=} \PY{n}{np}\PY{o}{.}\PY{n}{ndarray}\PY{p}{(}\PY{n}{shape}\PY{o}{=}\PY{p}{(}\PY{n+nb}{len}\PY{p}{(}\PY{n}{df\PYZus{}cols}\PY{p}{)}\PY{p}{,} \PY{l+m+mi}{2}\PY{p}{)}\PY{p}{)}
        \PY{k}{for} \PY{n}{i} \PY{o+ow}{in} \PY{n+nb}{range}\PY{p}{(}\PY{n+nb}{len}\PY{p}{(}\PY{n}{df\PYZus{}cols}\PY{p}{)}\PY{p}{)}\PY{p}{:}
            \PY{n}{attr} \PY{o}{=} \PY{n}{df\PYZus{}cols}\PY{p}{[}\PY{n}{i}\PY{p}{]}
            \PY{n}{data}\PY{p}{[}\PY{n}{i}\PY{p}{]} \PY{o}{=} \PY{n}{stats}\PY{o}{.}\PY{n}{pearsonr}\PY{p}{(}\PY{n}{wine\PYZus{}df}\PY{p}{[}\PY{n}{attr}\PY{p}{]}\PY{p}{,} \PY{n}{wine\PYZus{}df}\PY{p}{[}\PY{l+s+s1}{\PYZsq{}}\PY{l+s+s1}{quality}\PY{l+s+s1}{\PYZsq{}}\PY{p}{]}\PY{p}{)}
        \PY{n}{df\PYZus{}pearson} \PY{o}{=} \PY{n}{pd}\PY{o}{.}\PY{n}{DataFrame}\PY{p}{(}\PY{n}{data}\PY{p}{,} \PY{n}{index}\PY{o}{=}\PY{n}{df\PYZus{}cols}\PY{p}{,} \PY{n}{columns}\PY{o}{=}\PY{p}{[}\PY{l+s+s1}{\PYZsq{}}\PY{l+s+s1}{estimació}\PY{l+s+s1}{\PYZsq{}}\PY{p}{,} \PY{l+s+s1}{\PYZsq{}}\PY{l+s+s1}{p\PYZhy{}valor}\PY{l+s+s1}{\PYZsq{}}\PY{p}{]}\PY{p}{)}\PY{p}{[}\PY{p}{:}\PY{l+m+mi}{11}\PY{p}{]}
        \PY{n}{df\PYZus{}pearson}\PY{p}{[}\PY{l+s+s1}{\PYZsq{}}\PY{l+s+s1}{tmp\PYZus{}sort}\PY{l+s+s1}{\PYZsq{}}\PY{p}{]} \PY{o}{=} \PY{n}{df\PYZus{}pearson}\PY{p}{[}\PY{l+s+s1}{\PYZsq{}}\PY{l+s+s1}{estimació}\PY{l+s+s1}{\PYZsq{}}\PY{p}{]}\PY{o}{.}\PY{n}{abs}\PY{p}{(}\PY{p}{)}
        \PY{n}{df\PYZus{}pearson} \PY{o}{=} \PY{n}{df\PYZus{}pearson}\PY{o}{.}\PY{n}{sort\PYZus{}values}\PY{p}{(}\PY{n}{by}\PY{o}{=}\PY{l+s+s1}{\PYZsq{}}\PY{l+s+s1}{tmp\PYZus{}sort}\PY{l+s+s1}{\PYZsq{}}\PY{p}{,} \PY{n}{ascending}\PY{o}{=}\PY{k+kc}{False}\PY{p}{)}\PY{o}{.}\PY{n}{drop}\PY{p}{(}\PY{l+s+s1}{\PYZsq{}}\PY{l+s+s1}{tmp\PYZus{}sort}\PY{l+s+s1}{\PYZsq{}}\PY{p}{,} \PY{n}{axis}\PY{o}{=}\PY{l+m+mi}{1}\PY{p}{)}
        
        \PY{c+c1}{\PYZsh{} Format output}
        \PY{c+c1}{\PYZsh{} pprint(df\PYZus{}pearson[\PYZsq{}estimació\PYZsq{}].values)}
        \PY{n}{df\PYZus{}pearson}\PY{p}{[}\PY{l+s+s1}{\PYZsq{}}\PY{l+s+s1}{estimació}\PY{l+s+s1}{\PYZsq{}}\PY{p}{]} \PY{o}{=} \PY{p}{[}\PY{l+s+s2}{\PYZdq{}}\PY{l+s+si}{\PYZob{}0:.4f\PYZcb{}}\PY{l+s+s2}{\PYZdq{}}\PY{o}{.}\PY{n}{format}\PY{p}{(}\PY{n}{el}\PY{p}{)} \PY{k}{for} \PY{n}{el} \PY{o+ow}{in} \PY{n}{df\PYZus{}pearson}\PY{p}{[}\PY{l+s+s1}{\PYZsq{}}\PY{l+s+s1}{estimació}\PY{l+s+s1}{\PYZsq{}}\PY{p}{]}\PY{o}{.}\PY{n}{values}\PY{p}{]}
        \PY{n}{df\PYZus{}pearson}\PY{p}{[}\PY{l+s+s1}{\PYZsq{}}\PY{l+s+s1}{p\PYZhy{}valor}\PY{l+s+s1}{\PYZsq{}}\PY{p}{]} \PY{o}{=} \PY{p}{[}\PY{l+s+s2}{\PYZdq{}}\PY{l+s+si}{\PYZob{}0:.2e\PYZcb{}}\PY{l+s+s2}{\PYZdq{}}\PY{o}{.}\PY{n}{format}\PY{p}{(}\PY{n}{el}\PY{p}{)} \PY{k}{for} \PY{n}{el} \PY{o+ow}{in} \PY{n}{df\PYZus{}pearson}\PY{p}{[}\PY{l+s+s1}{\PYZsq{}}\PY{l+s+s1}{p\PYZhy{}valor}\PY{l+s+s1}{\PYZsq{}}\PY{p}{]}\PY{o}{.}\PY{n}{values}\PY{p}{]}
        \PY{n}{display}\PY{p}{(}\PY{n}{df\PYZus{}pearson}\PY{o}{.}\PY{n}{T}\PY{p}{)}
\end{Verbatim}


    
    \begin{verbatim}
            alcohol volatile acidity sulphates citric acid  \
estimació    0.4762          -0.3906    0.2514      0.2264   
p-valor    2.83e-91         2.05e-59  1.80e-24    4.99e-20   

          total sulfur dioxide   density chlorides fixed acidity        pH  \
estimació              -0.1851   -0.1749   -0.1289        0.1241   -0.0577   
p-valor               8.62e-14  1.87e-12  2.31e-07      6.50e-07  2.10e-02   

          free sulfur dioxide residual sugar  
estimació             -0.0507         0.0137  
p-valor              4.28e-02       5.83e-01  
    \end{verbatim}

    
    Hem ordenat els atributs dels que més impacten a l'hora de determinar la
qualitat als que menys. Tal i com podem observar, la qualitat augmenta
gairebé un 50\% segons el volum d'\textbf{alcohol} que tenim en el vi.
Per altra banda, el 40\% de quantitat d'\textbf{acidesa volàtil}
influeix negativament en la qualitat. És a dir, com més sàpiga a
vinagre, pitjor.

\begin{verbatim}
<p>
    Seguidament tenim que el 25% de la quantitat de **sulfats** augmenten la qualitat. Sembla obvi, doncs aquest component ajuda que el vi no s'oxidi i impedeix l'aparició de bacteris. Però per altra banda, paradoxalment, el **SO2 total i el lliure** resten qualitat al vi (un 18'5%  i un 5% de la seva quantitat, respectivament). 
</p>
<p>
    L'**àcid cítric** suma el 22'6% del seu valor i podem deduir que un vi que sigui més fresc afavoreix a obtenir una millor nota. Per altra banda, com menys dens és el vi -menys líquid- pitjor la seva qualitat, restant un 17'5% del valor de la **densitat**. La quantitat de **clorurs** també afecta negativament, restant quasi un 13% del seu valor: com més salat estigui el vi, pitjor qualitat tindrà. Per altra banda, el valor del **sucre** només suma un 1'4% del seu valor. No ens ha de resultar xocant doncs estem parlant de vins negres i aquests no acostumen a ser dolços (tret que siguin escumosos, però no és el cas d'aquests). 
</p>
<p>
    Darrerament, tenim que el **pH**, que resta un 5% del seu valor a la qualitat final. És a dir, com més àcid sigui un vi, pitjor encara que no compti molt. Però, per altra banda, l'**acidesa fixa** suma a la qualitat del vi un 12'4% del seu valor. 
</p>
<p>
    Pel que fa als p-valors, podem descartar tots aquells atributs que superin el llindar de 0.05. Només el sucre supera aquest llindar i, per tant, no el faríem servir en un model de regressió lineal.
</p>
\end{verbatim}

    \begin{Verbatim}[commandchars=\\\{\}]
{\color{incolor}In [{\color{incolor}9}]:} \PY{c+c1}{\PYZsh{} Correlació dels atributs}
        \PY{n}{pd}\PY{o}{.}\PY{n}{plotting}\PY{o}{.}\PY{n}{scatter\PYZus{}matrix}\PY{p}{(}\PY{n}{wine\PYZus{}df}\PY{o}{.}\PY{n}{iloc}\PY{p}{[}\PY{p}{:}\PY{p}{,} \PY{p}{:}\PY{l+m+mi}{11}\PY{p}{]}\PY{p}{,} \PY{n}{figsize}\PY{o}{=}\PY{p}{(}\PY{l+m+mi}{30}\PY{p}{,}\PY{l+m+mi}{15}\PY{p}{)}\PY{p}{)}
        \PY{n}{plt}\PY{o}{.}\PY{n}{show}\PY{p}{(}\PY{p}{)}
\end{Verbatim}


    \begin{center}
    \adjustimage{max size={0.9\linewidth}{0.9\paperheight}}{output_29_0.png}
    \end{center}
    { \hspace*{\fill} \\}
    
    En el gràfic de correlacions entre parells d'atributs -considerant que
són tots normals- podem veure clarament la relació directa que havíem
explicat al principi entre el SO2 lliure i total: com més tenim d'un,
més tenim de l'altre.

Per altra banda, tenim atributs que es relacionen fortament entre si.
Per exemple, com més àcid cítric, més acidesa fixa. Això ens fa pensar
que pujant aquest dos paràmetres obtindríem millor nota perquè, a més,
com més acidesa fixa, menys pH, que li resta qualitat. A més a més, com
més àcid cítric, menys acidesa volàtil i, per tant, menys sabor a
vinagre. Però tenim el problema que com més quantitat d'acidesa fixa,
derivada de l'àcid cítric, augmentem també la seva densitat, que va en
detriment de la qualitat. Com veiem, trobar els paràmetres
físico-químics equilibrats per treure un vi de molt bona qualitat és una
tasca força feixuga.

\begin{verbatim}
<p>
    A continuació, construirem un model de predicció. Serà una combinació d'arbres de decisió entrenats amb bagging (usant mostreig amb reemplaçament), també conegut com Random Forests. Hem escollit aquest classificador per diferents motius: curiositat acadèmica, aproximar-nos al treball amb eines d'entorns reals i, principalment, que és un model amb el coneixement accessible.
</p>
<p>
\end{verbatim}

Això vol dir que, a diferència d'altres models molt coneguts com les
xarxes neuronals, el model descriu perfectament perquè es comporta com
ho fa. Donat que la pregunta que volem resoldre és conèixer els factors
amb més pes per definir la qualitat del vi, el model \emph{Random
Forest} sembla el candidat idoni per aquesta tasca.

    \begin{Verbatim}[commandchars=\\\{\}]
{\color{incolor}In [{\color{incolor}10}]:} \PY{c+c1}{\PYZsh{}Inserció d\PYZsq{}atribut qualitatiu per la qualitat del vi}
         \PY{n}{CLASSES\PYZus{}NAMES} \PY{o}{=} \PY{p}{[}\PY{l+s+s1}{\PYZsq{}}\PY{l+s+s1}{Bad wine}\PY{l+s+s1}{\PYZsq{}}\PY{p}{,} \PY{l+s+s1}{\PYZsq{}}\PY{l+s+s1}{Good wine}\PY{l+s+s1}{\PYZsq{}}\PY{p}{]}
         \PY{n}{GOOD\PYZus{}THRESHOLD} \PY{o}{=} \PY{l+m+mi}{6}
         \PY{n}{raw\PYZus{}y} \PY{o}{=} \PY{n}{wine\PYZus{}df}\PY{p}{[}\PY{l+s+s1}{\PYZsq{}}\PY{l+s+s1}{quality}\PY{l+s+s1}{\PYZsq{}}\PY{p}{]} \PY{o}{\PYZgt{}}\PY{o}{=} \PY{n}{GOOD\PYZus{}THRESHOLD}
\end{Verbatim}


    \begin{Verbatim}[commandchars=\\\{\}]
{\color{incolor}In [{\color{incolor}11}]:} \PY{k}{def} \PY{n+nf}{encode\PYZus{}raw\PYZus{}y}\PY{p}{(}\PY{n}{raw\PYZus{}y}\PY{p}{)}\PY{p}{:}    
             \PY{n}{le} \PY{o}{=} \PY{n}{preprocessing}\PY{o}{.}\PY{n}{LabelEncoder}\PY{p}{(}\PY{p}{)}
             \PY{n}{y} \PY{o}{=} \PY{n}{le}\PY{o}{.}\PY{n}{fit\PYZus{}transform}\PY{p}{(}\PY{n}{raw\PYZus{}y}\PY{p}{)}
             \PY{k}{return} \PY{n}{y}
         
         \PY{k}{def} \PY{n+nf}{random\PYZus{}forest\PYZus{}preprocess\PYZus{}data}\PY{p}{(}\PY{n}{df}\PY{p}{,} \PY{n}{raw\PYZus{}y}\PY{p}{)}\PY{p}{:}
             \PY{c+c1}{\PYZsh{}Input data}
             \PY{n}{X} \PY{o}{=} \PY{n}{df}\PY{o}{.}\PY{n}{drop}\PY{p}{(}\PY{p}{[}\PY{l+s+s1}{\PYZsq{}}\PY{l+s+s1}{quality}\PY{l+s+s1}{\PYZsq{}}\PY{p}{]}\PY{p}{,} \PY{n}{axis}\PY{o}{=}\PY{l+m+mi}{1}\PY{p}{)}
             \PY{c+c1}{\PYZsh{}Labels}
             \PY{n}{y} \PY{o}{=} \PY{n}{encode\PYZus{}raw\PYZus{}y}\PY{p}{(}\PY{n}{raw\PYZus{}y}\PY{p}{)}
             \PY{k}{return} \PY{p}{(}\PY{n}{X}\PY{p}{,} \PY{n}{y}\PY{p}{)}
         
         \PY{k}{def} \PY{n+nf}{split\PYZus{}train\PYZus{}test}\PY{p}{(}\PY{n}{X}\PY{p}{,} \PY{n}{y}\PY{p}{)}\PY{p}{:}
             \PY{n}{X\PYZus{}train}\PY{p}{,} \PY{n}{X\PYZus{}test}\PY{p}{,} \PY{n}{y\PYZus{}train}\PY{p}{,} \PY{n}{y\PYZus{}test} \PY{o}{=} \PY{n}{train\PYZus{}test\PYZus{}split}\PY{p}{(}
                 \PY{n}{X}\PY{p}{,} \PY{n}{y}\PY{p}{,} \PY{n}{test\PYZus{}size}\PY{o}{=}\PY{l+m+mf}{0.15}\PY{p}{,} \PY{n}{random\PYZus{}state}\PY{o}{=}\PY{n}{RANDOM\PYZus{}SEED}
             \PY{p}{)}
             \PY{k}{return} \PY{p}{(}\PY{n}{X\PYZus{}train}\PY{p}{,} \PY{n}{X\PYZus{}test}\PY{p}{,} \PY{n}{y\PYZus{}train}\PY{p}{,} \PY{n}{y\PYZus{}test}\PY{p}{)}
         
         \PY{k}{def} \PY{n+nf}{display\PYZus{}precission}\PY{p}{(}\PY{n}{score}\PY{p}{)}\PY{p}{:}
             \PY{n+nb}{print}\PY{p}{(}\PY{l+s+s2}{\PYZdq{}}\PY{l+s+s2}{Precisió del model: }\PY{l+s+si}{\PYZob{}\PYZcb{}}\PY{l+s+s2}{ }\PY{l+s+s2}{\PYZpc{}}\PY{l+s+s2}{\PYZdq{}}\PY{o}{.}\PY{n}{format}\PY{p}{(}\PY{n+nb}{round}\PY{p}{(}\PY{n}{score}\PY{o}{*}\PY{l+m+mi}{100}\PY{p}{,} \PY{l+m+mi}{2}\PY{p}{)}\PY{p}{)}\PY{p}{)}
         
         \PY{p}{(}\PY{n}{X}\PY{p}{,} \PY{n}{y}\PY{p}{)} \PY{o}{=} \PY{n}{random\PYZus{}forest\PYZus{}preprocess\PYZus{}data}\PY{p}{(}\PY{n}{wine\PYZus{}df}\PY{p}{,} \PY{n}{raw\PYZus{}y}\PY{p}{)}
         \PY{p}{(}\PY{n}{X\PYZus{}train}\PY{p}{,} \PY{n}{X\PYZus{}test}\PY{p}{,} \PY{n}{y\PYZus{}train}\PY{p}{,} \PY{n}{y\PYZus{}test}\PY{p}{)} \PY{o}{=} \PY{n}{split\PYZus{}train\PYZus{}test}\PY{p}{(}\PY{n}{X}\PY{p}{,} \PY{n}{y}\PY{p}{)}
         
         \PY{n}{rfc\PYZus{}model} \PY{o}{=} \PY{n}{RandomForestClassifier}\PY{p}{(}\PY{n}{bootstrap}\PY{o}{=}\PY{k+kc}{True}\PY{p}{,} \PY{n}{n\PYZus{}estimators} \PY{o}{=} \PY{l+m+mi}{150}\PY{p}{,} \PY{n}{n\PYZus{}jobs}\PY{o}{=}\PY{l+m+mi}{8}\PY{p}{,} \PY{n}{random\PYZus{}state}\PY{o}{=}\PY{n}{RANDOM\PYZus{}SEED}\PY{p}{)}
         \PY{n}{rfc\PYZus{}model}\PY{o}{.}\PY{n}{fit}\PY{p}{(}\PY{n}{X\PYZus{}train}\PY{p}{,} \PY{n}{y\PYZus{}train}\PY{p}{)}
\end{Verbatim}


\begin{Verbatim}[commandchars=\\\{\}]
{\color{outcolor}Out[{\color{outcolor}11}]:} RandomForestClassifier(bootstrap=True, class\_weight=None, criterion='gini',
                     max\_depth=None, max\_features='auto', max\_leaf\_nodes=None,
                     min\_impurity\_decrease=0.0, min\_impurity\_split=None,
                     min\_samples\_leaf=1, min\_samples\_split=2,
                     min\_weight\_fraction\_leaf=0.0, n\_estimators=150, n\_jobs=8,
                     oob\_score=False, random\_state=2018, verbose=0,
                     warm\_start=False)
\end{Verbatim}
            
    \begin{Verbatim}[commandchars=\\\{\}]
{\color{incolor}In [{\color{incolor}12}]:} \PY{n}{rfc\PYZus{}score} \PY{o}{=} \PY{n}{rfc\PYZus{}model}\PY{o}{.}\PY{n}{score}\PY{p}{(}\PY{n}{X\PYZus{}test}\PY{p}{,} \PY{n}{y\PYZus{}test}\PY{p}{)}
         \PY{n}{display\PYZus{}precission}\PY{p}{(}\PY{n}{rfc\PYZus{}score}\PY{p}{)}
         \PY{c+c1}{\PYZsh{} pprint(rfc\PYZus{}model.feature\PYZus{}importances\PYZus{})}
         \PY{n}{importances\PYZus{}df} \PY{o}{=} \PY{n}{pd}\PY{o}{.}\PY{n}{DataFrame}\PY{p}{(}
             \PY{n}{rfc\PYZus{}model}\PY{o}{.}\PY{n}{feature\PYZus{}importances\PYZus{}}\PY{p}{,}
             \PY{n}{index}\PY{o}{=}\PY{n+nb}{list}\PY{p}{(}\PY{n}{X\PYZus{}test}\PY{p}{)}\PY{p}{,}
             \PY{n}{columns}\PY{o}{=}\PY{p}{[}\PY{l+s+s1}{\PYZsq{}}\PY{l+s+s1}{Importància}\PY{l+s+s1}{\PYZsq{}}\PY{p}{]}
         \PY{p}{)}
         \PY{n}{importances\PYZus{}df} \PY{o}{=} \PY{n}{importances\PYZus{}df}\PY{o}{.}\PY{n}{sort\PYZus{}values}\PY{p}{(}\PY{n}{by}\PY{o}{=}\PY{l+s+s2}{\PYZdq{}}\PY{l+s+s2}{Importància}\PY{l+s+s2}{\PYZdq{}}\PY{p}{,} \PY{n}{ascending}\PY{o}{=}\PY{k+kc}{False}\PY{p}{)}
         \PY{n}{display}\PY{p}{(}\PY{n}{importances\PYZus{}df}\PY{o}{.}\PY{n}{T}\PY{p}{)}
\end{Verbatim}


    \begin{Verbatim}[commandchars=\\\{\}]
Precisió del model: 78.75 \%

    \end{Verbatim}

    
    \begin{verbatim}
              alcohol  sulphates  volatile acidity  total sulfur dioxide  \
Importància  0.189099   0.126577          0.120986              0.099195   

              density  chlorides        pH  fixed acidity  citric acid  \
Importància  0.082641   0.071539  0.069156       0.066683     0.062752   

             free sulfur dioxide  residual sugar  
Importància             0.057955        0.053415  
    \end{verbatim}

    
    Com podem veure, la importància que atorga als atributs el nostre model
basat en boscos aleatoris (\emph{Random Forests}) són bastant semblants
als obtinguts en la correlació lineal, sent l'alcohol el de major pes i
el sucre residual el que menys. No hem de confondre els pesos que dóna
el \emph{Random Forest} als atributs (on la seva suma és 1) amb els
factors donats en la correlació lineal amb la qualitat.

Per altra banda, obtenim una precisió prou alta encara que una mica per
sota del llindar del 80\% al qual estem acostumats a donar per bo el
model. Hem de tenir en compte que un model complex com són els
\emph{Random Forests} millora la seva precisió com més gran és el seu
conjunt d'entrenament. Per tant, podem assegurar que utilitzant les
dades dels vins de collites d'altres anys -recordem que estem fent
servir només l'any 2009- millorarem la precisió de manera que aquest
model podrà determinar de forma més exacta la qualitat de vins futurs.

\begin{verbatim}
<p>
    Una vegada hem realitzat aquestes diferents proves i anàlisis, podem dir clarament que la característica amb més pes per determinar la qualitat del vi, sorprenentment, és el nivell d'alcohol.
</p>
</font>
\end{verbatim}

    \hypertarget{es-pot-inferir-a-partir-del-volum-dalcohol-la-qualitat-del-vi}{%
\subsubsection{Es pot inferir, a partir del volum d'alcohol, la qualitat
del
vi?}\label{es-pot-inferir-a-partir-del-volum-dalcohol-la-qualitat-del-vi}}

 Per tractar de respondre a aquesta pregunta, analitzarem la relació
entre les dimensions \texttt{alcohol\ \textasciitilde{}\ qualitat} i, si
escau, crearem un model de regressió lineal per tal de veure com es
relacionen. 

    \begin{Verbatim}[commandchars=\\\{\}]
{\color{incolor}In [{\color{incolor}13}]:} \PY{n}{alcohol\PYZus{}quality\PYZus{}df} \PY{o}{=} \PY{n}{wine\PYZus{}df}\PY{p}{[}\PY{p}{[}\PY{l+s+s1}{\PYZsq{}}\PY{l+s+s1}{alcohol}\PY{l+s+s1}{\PYZsq{}}\PY{p}{,} \PY{l+s+s1}{\PYZsq{}}\PY{l+s+s1}{quality}\PY{l+s+s1}{\PYZsq{}}\PY{p}{]}\PY{p}{]}
         
         \PY{n}{fig}\PY{p}{,} \PY{n}{ax} \PY{o}{=} \PY{n}{plt}\PY{o}{.}\PY{n}{subplots}\PY{p}{(}\PY{l+m+mi}{1}\PY{p}{,} \PY{l+m+mi}{1}\PY{p}{,} \PY{n}{figsize}\PY{o}{=}\PY{p}{(}\PY{l+m+mi}{25}\PY{p}{,} \PY{l+m+mi}{7}\PY{p}{)}\PY{p}{)}
         \PY{n}{ax}\PY{o}{.}\PY{n}{scatter}\PY{p}{(}\PY{n}{alcohol\PYZus{}quality\PYZus{}df}\PY{p}{[}\PY{l+s+s1}{\PYZsq{}}\PY{l+s+s1}{alcohol}\PY{l+s+s1}{\PYZsq{}}\PY{p}{]}\PY{p}{,} \PY{n}{alcohol\PYZus{}quality\PYZus{}df}\PY{p}{[}\PY{l+s+s1}{\PYZsq{}}\PY{l+s+s1}{quality}\PY{l+s+s1}{\PYZsq{}}\PY{p}{]}\PY{p}{)}
         \PY{n}{ax}\PY{o}{.}\PY{n}{set\PYZus{}xlabel}\PY{p}{(}\PY{l+s+s1}{\PYZsq{}}\PY{l+s+s1}{Graduació alcoholica}\PY{l+s+s1}{\PYZsq{}}\PY{p}{)}
         \PY{n}{ax}\PY{o}{.}\PY{n}{set\PYZus{}ylabel}\PY{p}{(}\PY{l+s+s1}{\PYZsq{}}\PY{l+s+s1}{Qualitat}\PY{l+s+s1}{\PYZsq{}}\PY{p}{)}
         \PY{n}{plt}\PY{o}{.}\PY{n}{show}\PY{p}{(}\PY{p}{)}
\end{Verbatim}


    \begin{center}
    \adjustimage{max size={0.9\linewidth}{0.9\paperheight}}{output_36_0.png}
    \end{center}
    { \hspace*{\fill} \\}
    
    Com podem observar, en aquest cas és obvi que sobre aquestes dades no es
pot aplicar un model de regressió lineal, ja que el resultat serà de
molt baixa qualitat. Això és degut al fet que la variable qualitat és
discreta, i no continua. Tanmateix, s'observa una mena de diagonal que
podria fer-nos intuir que la qualitat augmenta així com ho fa la
graduació alcohòlica. En cas de voler realitzar una regressió lineal amb
les eines usades en aquest exercici, s'hauria d'implementar un codi molt
semblant al descrit a la documentació de sklearn.

Creiem que, com que ja disposem d'un model per inferir la qualitat del
vi, el que podríem fer és tractar d'entrenar el model només amb la
graduació alcohòlica i analitzar-ne el comportament.

    \begin{Verbatim}[commandchars=\\\{\}]
{\color{incolor}In [{\color{incolor}14}]:} \PY{p}{(}\PY{n}{X\PYZus{}alcohol\PYZus{}train}\PY{p}{,} \PY{n}{X\PYZus{}alcohol\PYZus{}test}\PY{p}{,} \PY{n}{y\PYZus{}alcohol\PYZus{}train}\PY{p}{,} \PY{n}{y\PYZus{}alcohol\PYZus{}test}\PY{p}{)} \PY{o}{=} \PY{n}{split\PYZus{}train\PYZus{}test}\PY{p}{(}\PY{n}{alcohol\PYZus{}quality\PYZus{}df}\PY{p}{[}\PY{l+s+s1}{\PYZsq{}}\PY{l+s+s1}{alcohol}\PY{l+s+s1}{\PYZsq{}}\PY{p}{]}\PY{p}{,} \PY{n}{encode\PYZus{}raw\PYZus{}y}\PY{p}{(}\PY{n}{raw\PYZus{}y}\PY{p}{)}\PY{p}{)}
         \PY{n}{X\PYZus{}alcohol\PYZus{}train} \PY{o}{=} \PY{n}{X\PYZus{}alcohol\PYZus{}train}\PY{o}{.}\PY{n}{values}\PY{o}{.}\PY{n}{reshape}\PY{p}{(}\PY{o}{\PYZhy{}}\PY{l+m+mi}{1}\PY{p}{,} \PY{l+m+mi}{1}\PY{p}{)}
         \PY{n}{X\PYZus{}alcohol\PYZus{}test} \PY{o}{=} \PY{n}{X\PYZus{}alcohol\PYZus{}test}\PY{o}{.}\PY{n}{values}\PY{o}{.}\PY{n}{reshape}\PY{p}{(}\PY{o}{\PYZhy{}}\PY{l+m+mi}{1}\PY{p}{,} \PY{l+m+mi}{1}\PY{p}{)}
         
         \PY{n}{rfc\PYZus{}alcohol\PYZus{}model} \PY{o}{=} \PY{n}{RandomForestClassifier}\PY{p}{(}\PY{n}{bootstrap}\PY{o}{=}\PY{k+kc}{True}\PY{p}{,} \PY{n}{n\PYZus{}estimators} \PY{o}{=} \PY{l+m+mi}{150}\PY{p}{,} \PY{n}{n\PYZus{}jobs}\PY{o}{=}\PY{l+m+mi}{8}\PY{p}{,} \PY{n}{random\PYZus{}state}\PY{o}{=}\PY{n}{RANDOM\PYZus{}SEED}\PY{p}{)}
         \PY{n}{rfc\PYZus{}alcohol\PYZus{}model}\PY{o}{.}\PY{n}{fit}\PY{p}{(}\PY{n}{X\PYZus{}alcohol\PYZus{}train}\PY{p}{,} \PY{n}{y\PYZus{}alcohol\PYZus{}train}\PY{p}{)}
         \PY{n}{rfc\PYZus{}alcohol\PYZus{}score} \PY{o}{=} \PY{n}{rfc\PYZus{}alcohol\PYZus{}model}\PY{o}{.}\PY{n}{score}\PY{p}{(}\PY{n}{X\PYZus{}alcohol\PYZus{}test}\PY{p}{,} \PY{n}{y\PYZus{}alcohol\PYZus{}test}\PY{p}{)}
         
         \PY{n}{display\PYZus{}precission}\PY{p}{(}\PY{n}{rfc\PYZus{}alcohol\PYZus{}score}\PY{p}{)}
\end{Verbatim}


    \begin{Verbatim}[commandchars=\\\{\}]
Precisió del model: 67.08 \%

    \end{Verbatim}

    Notem com la precisió ha caigut considerablement, com era d'esperar. A
més, també hem de tenir present que encara que 67\% no sembli un valor
especialment baix, no és gaire vàlid. Donat que estam intentant modelar
sobre dues classes, un classificador aleatori ens haurà de donar un 50\%
de precisió, així que aquest és el nostre punt de partida.

Com a resposta a la pregunta original, que es demana ``Es pot inferir, a
partir del volum d'alcohol, la qualitat del vi?'', amb els resultats
obtinguts, tot i no ser una precisió menyspreable, consideram que el
model no és suficientment vàlid com per determinar la qualitat del vi
només coneixent-ne el volum d'alcohol.

    \hypertarget{representacio-dels-resultats-a-partir-de-taules-i-grafiques.}{%
\subsection{5. Representació dels resultats a partir de taules i
gràfiques.}\label{representacio-dels-resultats-a-partir-de-taules-i-grafiques.}}

Durant les diferents etapes de neteja i anàlisi hem mostrat les taules i
gràfiques adients per entendre el comportament de les accions. Tot i
això, a continuació inserim una sèrie de gràfiques addicionals que
permeten obtenir encara més coneixement del conjunt de dades.

    \hypertarget{anuxe0lisi-de-normalitat}{%
\subsubsection{Anàlisi de normalitat}\label{anuxe0lisi-de-normalitat}}

    \begin{Verbatim}[commandchars=\\\{\}]
{\color{incolor}In [{\color{incolor}15}]:} \PY{k}{def} \PY{n+nf}{plot\PYZus{}qq}\PY{p}{(}\PY{n}{keys}\PY{p}{)}\PY{p}{:}
             \PY{k}{if} \PY{n}{keys} \PY{o+ow}{is} \PY{k+kc}{None}\PY{p}{:}
                 \PY{n}{keys} \PY{o}{=} \PY{n+nb}{list}\PY{p}{(}\PY{n}{wine\PYZus{}df}\PY{p}{)}
                 
             \PY{n}{cols} \PY{o}{=} \PY{l+m+mi}{4}
             \PY{n}{rows} \PY{o}{=} \PY{n}{math}\PY{o}{.}\PY{n}{ceil}\PY{p}{(}\PY{n+nb}{len}\PY{p}{(}\PY{n}{keys}\PY{p}{)}\PY{o}{/}\PY{n}{cols}\PY{p}{)}
             \PY{n}{plt}\PY{o}{.}\PY{n}{figure}\PY{p}{(}\PY{n}{figsize}\PY{o}{=}\PY{p}{(}\PY{l+m+mi}{25}\PY{p}{,} \PY{l+m+mi}{7}\PY{p}{)}\PY{p}{)}
             \PY{k}{for} \PY{n}{i}\PY{p}{,} \PY{n}{attr\PYZus{}key} \PY{o+ow}{in} \PY{n+nb}{enumerate}\PY{p}{(}\PY{n}{keys}\PY{p}{)}\PY{p}{:}
                 \PY{n}{plt}\PY{o}{.}\PY{n}{subplot}\PY{p}{(}\PY{n}{rows}\PY{p}{,} \PY{n}{cols}\PY{p}{,} \PY{n}{i}\PY{o}{+}\PY{l+m+mi}{1}\PY{p}{)}
                 \PY{n}{attr} \PY{o}{=} \PY{n}{wine\PYZus{}df}\PY{p}{[}\PY{n}{attr\PYZus{}key}\PY{p}{]}
                 \PY{n}{stats}\PY{o}{.}\PY{n}{probplot}\PY{p}{(}\PY{n}{attr}\PY{p}{,} \PY{n}{dist}\PY{o}{=}\PY{l+s+s2}{\PYZdq{}}\PY{l+s+s2}{norm}\PY{l+s+s2}{\PYZdq{}}\PY{p}{,} \PY{n}{plot}\PY{o}{=}\PY{n}{plt}\PY{p}{)}
             \PY{n}{plt}\PY{o}{.}\PY{n}{show}\PY{p}{(}\PY{p}{)}
         
         \PY{n}{plot\PYZus{}qq}\PY{p}{(}\PY{p}{[}\PY{l+s+s1}{\PYZsq{}}\PY{l+s+s1}{alcohol}\PY{l+s+s1}{\PYZsq{}}\PY{p}{,} \PY{l+s+s1}{\PYZsq{}}\PY{l+s+s1}{density}\PY{l+s+s1}{\PYZsq{}}\PY{p}{,} \PY{l+s+s1}{\PYZsq{}}\PY{l+s+s1}{quality}\PY{l+s+s1}{\PYZsq{}}\PY{p}{]}\PY{p}{)}
\end{Verbatim}


    \begin{center}
    \adjustimage{max size={0.9\linewidth}{0.9\paperheight}}{output_42_0.png}
    \end{center}
    { \hspace*{\fill} \\}
    
    La gràfica Q-Q és molt usada per comprovar la normalitat d'unes sèries
de dades. Tanmateix, amb la informació trobada a l'apartat 4.2 ha sigut
suficient per veure la normalitat de les dades, així que aquesta
implementació ha quedat relegada a una mena d'apèndix. Com més
s'aproximen els quantils teòrics a la bisectriu, més normal és la
distribució. Fins i tot, podem veure que la distribució de l'atribut de
la qualitat, tot i ser discret, és distribueix de forma equitativa al
llarg de la bisectriu. 

    \hypertarget{arbre-de-decisiuxf3}{%
\subsubsection{Arbre de decisió}\label{arbre-de-decisiuxf3}}

    Seleccionat una mostra aleatòria dels 250 arbres del
\emph{RandomForest}, observam com l'arbre de decisió és relativament
complex. Tot i això, renderitzar-lo és una bona manera de seguir les
decisions que aquests pren en funció de les dades d'entrada. Per
simplificar el model visualment, podem generar un nou
\emph{RandomForest} definit una profunditat màxima dels arbres. Així,
òbviament perdrem qualitat de predicció, però és un bon exercici
acadèmic.

    \begin{Verbatim}[commandchars=\\\{\}]
{\color{incolor}In [{\color{incolor}16}]:} \PY{k}{def} \PY{n+nf}{render\PYZus{}tree}\PY{p}{(}\PY{n}{tree}\PY{p}{)}\PY{p}{:}
             \PY{n}{dot\PYZus{}data} \PY{o}{=} \PY{n}{StringIO}\PY{p}{(}\PY{p}{)}
             \PY{n}{export\PYZus{}graphviz}\PY{p}{(}\PY{n}{tree}\PY{p}{,} \PY{n}{out\PYZus{}file}\PY{o}{=}\PY{n}{dot\PYZus{}data}\PY{p}{,}  
                             \PY{n}{filled}\PY{o}{=}\PY{k+kc}{True}\PY{p}{,} \PY{n}{rounded}\PY{o}{=}\PY{k+kc}{True}\PY{p}{,}
                             \PY{n}{class\PYZus{}names}\PY{o}{=}\PY{n}{CLASSES\PYZus{}NAMES}\PY{p}{,}
                             \PY{n}{feature\PYZus{}names}\PY{o}{=}\PY{n+nb}{list}\PY{p}{(}\PY{n}{wine\PYZus{}df}\PY{o}{.}\PY{n}{drop}\PY{p}{(}\PY{n}{columns}\PY{o}{=}\PY{l+s+s1}{\PYZsq{}}\PY{l+s+s1}{quality}\PY{l+s+s1}{\PYZsq{}}\PY{p}{)}\PY{p}{)}\PY{p}{,}
                             \PY{n}{special\PYZus{}characters}\PY{o}{=}\PY{k+kc}{True}\PY{p}{)}
         
             \PY{n}{graph} \PY{o}{=} \PY{n}{pydotplus}\PY{o}{.}\PY{n}{graph\PYZus{}from\PYZus{}dot\PYZus{}data}\PY{p}{(}\PY{n}{dot\PYZus{}data}\PY{o}{.}\PY{n}{getvalue}\PY{p}{(}\PY{p}{)}\PY{p}{)}  
             \PY{n}{display}\PY{p}{(}\PY{n}{Image}\PY{p}{(}\PY{n}{graph}\PY{o}{.}\PY{n}{create\PYZus{}png}\PY{p}{(}\PY{p}{)}\PY{p}{)}\PY{p}{)}
\end{Verbatim}


    \begin{Verbatim}[commandchars=\\\{\}]
{\color{incolor}In [{\color{incolor}17}]:} \PY{c+c1}{\PYZsh{} Render random forest sample tree}
         \PY{n}{sample\PYZus{}tree} \PY{o}{=} \PY{n}{np}\PY{o}{.}\PY{n}{random}\PY{o}{.}\PY{n}{choice}\PY{p}{(}\PY{n}{rfc\PYZus{}model}\PY{o}{.}\PY{n}{estimators\PYZus{}}\PY{p}{)}
         \PY{n}{render\PYZus{}tree}\PY{p}{(}\PY{n}{sample\PYZus{}tree}\PY{p}{)}
\end{Verbatim}


    \begin{center}
    \adjustimage{max size={0.9\linewidth}{0.9\paperheight}}{output_47_0.png}
    \end{center}
    { \hspace*{\fill} \\}
    
    \begin{Verbatim}[commandchars=\\\{\}]
{\color{incolor}In [{\color{incolor}18}]:} \PY{n}{rfc\PYZus{}model\PYZus{}basic\PYZus{}trees} \PY{o}{=} \PY{n}{RandomForestClassifier}\PY{p}{(}\PY{n}{max\PYZus{}depth}\PY{o}{=}\PY{l+m+mi}{3}\PY{p}{,} \PY{n}{bootstrap}\PY{o}{=}\PY{k+kc}{True}\PY{p}{,} \PY{n}{n\PYZus{}estimators} \PY{o}{=} \PY{l+m+mi}{150}\PY{p}{,} \PY{n}{n\PYZus{}jobs}\PY{o}{=}\PY{l+m+mi}{8}\PY{p}{,} \PY{n}{random\PYZus{}state}\PY{o}{=}\PY{n}{RANDOM\PYZus{}SEED}\PY{p}{)}
         \PY{n}{rfc\PYZus{}model\PYZus{}basic\PYZus{}trees}\PY{o}{.}\PY{n}{fit}\PY{p}{(}\PY{n}{X\PYZus{}train}\PY{p}{,} \PY{n}{y\PYZus{}train}\PY{p}{)}
         
         \PY{n}{sample\PYZus{}tree} \PY{o}{=} \PY{n}{np}\PY{o}{.}\PY{n}{random}\PY{o}{.}\PY{n}{choice}\PY{p}{(}\PY{n}{rfc\PYZus{}model\PYZus{}basic\PYZus{}trees}\PY{o}{.}\PY{n}{estimators\PYZus{}}\PY{p}{)}
         \PY{n}{render\PYZus{}tree}\PY{p}{(}\PY{n}{sample\PYZus{}tree}\PY{p}{)}
         
         \PY{n+nb}{print}\PY{p}{(}\PY{l+s+s2}{\PYZdq{}}\PY{l+s+s2}{Precisió del model: }\PY{l+s+si}{\PYZob{}\PYZcb{}}\PY{l+s+s2}{ }\PY{l+s+s2}{\PYZpc{}}\PY{l+s+s2}{\PYZdq{}}\PY{o}{.}\PY{n}{format}\PY{p}{(}\PY{n+nb}{round}\PY{p}{(}\PY{n}{rfc\PYZus{}model\PYZus{}basic\PYZus{}trees}\PY{o}{.}\PY{n}{score}\PY{p}{(}\PY{n}{X\PYZus{}test}\PY{p}{,} \PY{n}{y\PYZus{}test}\PY{p}{)}\PY{o}{*}\PY{l+m+mi}{100}\PY{p}{,} \PY{l+m+mi}{2}\PY{p}{)}\PY{p}{)}\PY{p}{)}
\end{Verbatim}


    \begin{center}
    \adjustimage{max size={0.9\linewidth}{0.9\paperheight}}{output_48_0.png}
    \end{center}
    { \hspace*{\fill} \\}
    
    \begin{Verbatim}[commandchars=\\\{\}]
Precisió del model: 71.25 \%

    \end{Verbatim}

    Ara sí, podem avaluar i seguir fàcilment les decisions que pren aquest
nou model. Sorprenentment observem com la precisió del conjunt del
\emph{Random Forest} no ha baixat gaire, i es situa en un 71.25\%, només
7.5 punts per sota del model original.

    \hypertarget{matriu-de-confusiuxf3}{%
\subsubsection{Matriu de confusió}\label{matriu-de-confusiuxf3}}

Una altra eina molt útil a l'hora d'avaluar models de regressió és la
matriu de confusió. Aquesta ens mostra, per cada classe, el nombre de
Vertaders positius (\emph{TP}), Falsos postius (\emph{FP}), Vertaders
Negatius (\emph{TN}) i Falsos negatius (\emph{FN}).

    \begin{Verbatim}[commandchars=\\\{\}]
{\color{incolor}In [{\color{incolor}19}]:} \PY{c+c1}{\PYZsh{} http://scikit\PYZhy{}learn.org/stable/auto\PYZus{}examples/model\PYZus{}selection/plot\PYZus{}confusion\PYZus{}matrix.html}
         \PY{k}{def} \PY{n+nf}{plot\PYZus{}confusion\PYZus{}matrix}\PY{p}{(}\PY{n}{cm}\PY{p}{,} \PY{n}{classes}\PY{p}{,}
                                   \PY{n}{normalize}\PY{o}{=}\PY{k+kc}{False}\PY{p}{,}
                                   \PY{n}{title}\PY{o}{=}\PY{l+s+s1}{\PYZsq{}}\PY{l+s+s1}{Matriu de confusió}\PY{l+s+s1}{\PYZsq{}}\PY{p}{,}
                                   \PY{n}{cmap}\PY{o}{=}\PY{n}{plt}\PY{o}{.}\PY{n}{cm}\PY{o}{.}\PY{n}{Blues}\PY{p}{)}\PY{p}{:}
             \PY{l+s+sd}{\PYZdq{}\PYZdq{}\PYZdq{}}
         \PY{l+s+sd}{    This function prints and plots the confusion matrix.}
         \PY{l+s+sd}{    Normalization can be applied by setting `normalize=True`.}
         \PY{l+s+sd}{    \PYZdq{}\PYZdq{}\PYZdq{}}
             \PY{k}{if} \PY{n}{normalize}\PY{p}{:}
                 \PY{n}{cm} \PY{o}{=} \PY{n}{cm}\PY{o}{.}\PY{n}{astype}\PY{p}{(}\PY{l+s+s1}{\PYZsq{}}\PY{l+s+s1}{float}\PY{l+s+s1}{\PYZsq{}}\PY{p}{)} \PY{o}{/} \PY{n}{cm}\PY{o}{.}\PY{n}{sum}\PY{p}{(}\PY{n}{axis}\PY{o}{=}\PY{l+m+mi}{1}\PY{p}{)}\PY{p}{[}\PY{p}{:}\PY{p}{,} \PY{n}{np}\PY{o}{.}\PY{n}{newaxis}\PY{p}{]}
         
             \PY{n}{plt}\PY{o}{.}\PY{n}{imshow}\PY{p}{(}\PY{n}{cm}\PY{p}{,} \PY{n}{interpolation}\PY{o}{=}\PY{l+s+s1}{\PYZsq{}}\PY{l+s+s1}{nearest}\PY{l+s+s1}{\PYZsq{}}\PY{p}{,} \PY{n}{cmap}\PY{o}{=}\PY{n}{cmap}\PY{p}{)}
             \PY{n}{plt}\PY{o}{.}\PY{n}{title}\PY{p}{(}\PY{n}{title}\PY{p}{)}
             \PY{n}{plt}\PY{o}{.}\PY{n}{colorbar}\PY{p}{(}\PY{p}{)}
             \PY{n}{tick\PYZus{}marks} \PY{o}{=} \PY{n}{np}\PY{o}{.}\PY{n}{arange}\PY{p}{(}\PY{n+nb}{len}\PY{p}{(}\PY{n}{classes}\PY{p}{)}\PY{p}{)}
             \PY{n}{plt}\PY{o}{.}\PY{n}{xticks}\PY{p}{(}\PY{n}{tick\PYZus{}marks}\PY{p}{,} \PY{n}{classes}\PY{p}{,} \PY{n}{rotation}\PY{o}{=}\PY{l+m+mi}{45}\PY{p}{)}
             \PY{n}{plt}\PY{o}{.}\PY{n}{yticks}\PY{p}{(}\PY{n}{tick\PYZus{}marks}\PY{p}{,} \PY{n}{classes}\PY{p}{)}
         
             \PY{n}{fmt} \PY{o}{=} \PY{l+s+s1}{\PYZsq{}}\PY{l+s+s1}{.2f}\PY{l+s+s1}{\PYZsq{}} \PY{k}{if} \PY{n}{normalize} \PY{k}{else} \PY{l+s+s1}{\PYZsq{}}\PY{l+s+s1}{d}\PY{l+s+s1}{\PYZsq{}}
             \PY{n}{thresh} \PY{o}{=} \PY{n}{cm}\PY{o}{.}\PY{n}{max}\PY{p}{(}\PY{p}{)} \PY{o}{/} \PY{l+m+mf}{2.}
             \PY{k}{for} \PY{n}{i}\PY{p}{,} \PY{n}{j} \PY{o+ow}{in} \PY{n}{itertools}\PY{o}{.}\PY{n}{product}\PY{p}{(}\PY{n+nb}{range}\PY{p}{(}\PY{n}{cm}\PY{o}{.}\PY{n}{shape}\PY{p}{[}\PY{l+m+mi}{0}\PY{p}{]}\PY{p}{)}\PY{p}{,} \PY{n+nb}{range}\PY{p}{(}\PY{n}{cm}\PY{o}{.}\PY{n}{shape}\PY{p}{[}\PY{l+m+mi}{1}\PY{p}{]}\PY{p}{)}\PY{p}{)}\PY{p}{:}
                 \PY{n}{plt}\PY{o}{.}\PY{n}{text}\PY{p}{(}\PY{n}{j}\PY{p}{,} \PY{n}{i}\PY{p}{,} \PY{n+nb}{format}\PY{p}{(}\PY{n}{cm}\PY{p}{[}\PY{n}{i}\PY{p}{,} \PY{n}{j}\PY{p}{]}\PY{p}{,} \PY{n}{fmt}\PY{p}{)}\PY{p}{,}
                          \PY{n}{horizontalalignment}\PY{o}{=}\PY{l+s+s2}{\PYZdq{}}\PY{l+s+s2}{center}\PY{l+s+s2}{\PYZdq{}}\PY{p}{,}
                          \PY{n}{color}\PY{o}{=}\PY{l+s+s2}{\PYZdq{}}\PY{l+s+s2}{white}\PY{l+s+s2}{\PYZdq{}} \PY{k}{if} \PY{n}{cm}\PY{p}{[}\PY{n}{i}\PY{p}{,} \PY{n}{j}\PY{p}{]} \PY{o}{\PYZgt{}} \PY{n}{thresh} \PY{k}{else} \PY{l+s+s2}{\PYZdq{}}\PY{l+s+s2}{black}\PY{l+s+s2}{\PYZdq{}}\PY{p}{)}
         
             \PY{n}{plt}\PY{o}{.}\PY{n}{tight\PYZus{}layout}\PY{p}{(}\PY{p}{)}
             \PY{n}{plt}\PY{o}{.}\PY{n}{ylabel}\PY{p}{(}\PY{l+s+s1}{\PYZsq{}}\PY{l+s+s1}{True label}\PY{l+s+s1}{\PYZsq{}}\PY{p}{)}
             \PY{n}{plt}\PY{o}{.}\PY{n}{xlabel}\PY{p}{(}\PY{l+s+s1}{\PYZsq{}}\PY{l+s+s1}{Predicted label}\PY{l+s+s1}{\PYZsq{}}\PY{p}{)}
\end{Verbatim}


    \begin{Verbatim}[commandchars=\\\{\}]
{\color{incolor}In [{\color{incolor}20}]:} \PY{n}{y\PYZus{}pred} \PY{o}{=} \PY{n}{rfc\PYZus{}model}\PY{o}{.}\PY{n}{predict}\PY{p}{(}\PY{n}{X\PYZus{}test}\PY{p}{)}
         \PY{n}{cnf\PYZus{}matrix} \PY{o}{=} \PY{n}{confusion\PYZus{}matrix}\PY{p}{(}\PY{n}{y\PYZus{}test}\PY{p}{,} \PY{n}{y\PYZus{}pred}\PY{p}{)}
         
         \PY{n}{plot\PYZus{}confusion\PYZus{}matrix}\PY{p}{(}\PY{n}{cnf\PYZus{}matrix}\PY{p}{,} \PY{n}{CLASSES\PYZus{}NAMES}\PY{p}{,} \PY{n}{normalize}\PY{o}{=}\PY{k+kc}{True}\PY{p}{)}
\end{Verbatim}


    \begin{center}
    \adjustimage{max size={0.9\linewidth}{0.9\paperheight}}{output_52_0.png}
    \end{center}
    { \hspace*{\fill} \\}
    
     Podem observar com les prediccions, tant encertades com errònies estan
molt equilibrades, així que no sembla que cap de les dues classes sigui
un factor determinant a l'hora d'usar el model generat. Això és degut al
fet que el \emph{threshold} definit per dir si un vi és bo o dolent és
molt proper a la mitjana i coincideix amb la mediana.

    \hypertarget{resolucio-del-problema.-a-partir-dels-resultats-obtinguts-quines-son-les-conclusions-els-resultats-permeten-respondre-al-problema}{%
\subsection{6. Resolució del problema. A partir dels resultats
obtinguts, quines són les conclusions? Els resultats permeten respondre
al
problema?}\label{resolucio-del-problema.-a-partir-dels-resultats-obtinguts-quines-son-les-conclusions-els-resultats-permeten-respondre-al-problema}}

    Com hem pogut veure al llarg de tot el document, tots els atributs
presents ajuden a determinar la qualitat del vi. Tot i això, hem
observat com atributs com els atributs d'alcohol, sulfats o l'acidesa
volàtil tenen més pes a la qualitat del vi. A més, aquesta idea ha estat
doblement validada, ja que hem descobert aquesta informació mitjançant
l'anàlisi de correlacions i mitjançant el \emph{Random Forest}.

Per altre costat, hem pogut veure que podem predir la qualitat del vi a
partir de l'alcohol amb una precisió del 67\%. Tot i que hi ha estudis
que donen per bones precisions per sobre del 60\%, el nostre llindar és
a partir del 80\% de precisió per considerar un model com fiable. Per
tant, s'hauran de recórrer a més atributs per poder predir la qualitat
de forma lineal.

\begin{verbatim}
<p>
    Per altra banda, gràcies a respondre les preguntes originals, també hem pogut construir un model de predicció bastant robust que pot classificar els vins entre bons i, diguem-ne, no tan bons, amb una fiabilitat de gairebé el 79%. Com ja s'ha comentat, si a aquest model li donem dades d'anys posteriors a l'any 2009 per entrenar-se, obtindrem un classificador molt fiable per determinar la qualitat dels vins d'enguany.
</p>
</font>
\end{verbatim}

    \hypertarget{codi-cal-adjuntar-el-codi-preferiblement-en-r-amb-el-que-sha-realitzat-la-neteja-analisi-i-representacio-de-les-dades.-si-ho-preferiu-tambe-podeu-treballar-en-python}{%
\subsection{7. Codi: Cal adjuntar el codi, preferiblement en R, amb el
que s'ha realitzat la neteja, anàlisi i representació de les dades. Si
ho preferiu, també podeu treballar en
Python}\label{codi-cal-adjuntar-el-codi-preferiblement-en-r-amb-el-que-sha-realitzat-la-neteja-analisi-i-representacio-de-les-dades.-si-ho-preferiu-tambe-podeu-treballar-en-python}}

     Tot el codi emprat en aquesta pràctica es pot trobar en les diferents
preguntes en les cel·les precedides amb \texttt{In\ {[}\ {]}} i es poden
executar des de jupyter o qualsevol terminal interactiva de Python. 

    \hypertarget{expotaciuxf3-del-fitxer-amb-les-dades-tractades}{%
\subsection{8. Expotació del fitxer amb les dades
tractades}\label{expotaciuxf3-del-fitxer-amb-les-dades-tractades}}

 Donat que no s'ha hagut d'aplicar cap modificació a les dades originals
i s'han fet servir totes les variables, podem dir que el \emph{dataset}
amb les dades tractades és el mateix que l'original. Tot i això, per
demostrar l'ús de l'eina d'exportaicó, adjuntarem la classe que defineix
si el vi és bo o dolent com una columna més, i procedirem a exportar el
\emph{dataset}. 

    \begin{Verbatim}[commandchars=\\\{\}]
{\color{incolor}In [{\color{incolor}21}]:} \PY{n}{TREATED\PYZus{}CSV\PYZus{}PATH} \PY{o}{=} \PY{l+s+s1}{\PYZsq{}}\PY{l+s+s1}{dataset\PYZhy{}tractat.csv}\PY{l+s+s1}{\PYZsq{}}
         
         \PY{n}{exportable\PYZus{}df} \PY{o}{=} \PY{n}{wine\PYZus{}df}\PY{o}{.}\PY{n}{copy}\PY{p}{(}\PY{p}{)}
         \PY{n}{exportable\PYZus{}df}\PY{p}{[}\PY{l+s+s1}{\PYZsq{}}\PY{l+s+s1}{label}\PY{l+s+s1}{\PYZsq{}}\PY{p}{]} \PY{o}{=} \PY{n}{raw\PYZus{}y}
         \PY{n}{exportable\PYZus{}df}\PY{p}{[}\PY{l+s+s1}{\PYZsq{}}\PY{l+s+s1}{label}\PY{l+s+s1}{\PYZsq{}}\PY{p}{]} \PY{o}{=} \PY{n}{exportable\PYZus{}df}\PY{p}{[}\PY{l+s+s1}{\PYZsq{}}\PY{l+s+s1}{label}\PY{l+s+s1}{\PYZsq{}}\PY{p}{]}\PY{o}{.}\PY{n}{map}\PY{p}{(}\PY{p}{\PYZob{}}\PY{k+kc}{True}\PY{p}{:} \PY{l+s+s1}{\PYZsq{}}\PY{l+s+s1}{GoodWine}\PY{l+s+s1}{\PYZsq{}}\PY{p}{,} \PY{k+kc}{False}\PY{p}{:} \PY{l+s+s1}{\PYZsq{}}\PY{l+s+s1}{BadWine}\PY{l+s+s1}{\PYZsq{}}\PY{p}{\PYZcb{}}\PY{p}{)}
         
         \PY{n}{exportable\PYZus{}df}\PY{o}{.}\PY{n}{to\PYZus{}csv}\PY{p}{(}\PY{n}{TREATED\PYZus{}CSV\PYZus{}PATH}\PY{p}{,} \PY{n}{sep}\PY{o}{=}\PY{l+s+s1}{\PYZsq{}}\PY{l+s+s1}{;}\PY{l+s+s1}{\PYZsq{}}\PY{p}{,} \PY{n}{index}\PY{o}{=}\PY{k+kc}{False}\PY{p}{)}
\end{Verbatim}



    % Add a bibliography block to the postdoc
    
    
    
    \end{document}
